%%% Copyright (C) 2018 Vincent Goulet
%%%
%%% Ce fichier fait partie du projet «Méthodes numériques en actuariat»
%%% http://github.com/vigou3/methodes-numeriques-en-actuariat
%%%
%%% Cette création est mise à disposition selon le contrat
%%% Attribution-Partage dans les mêmes conditions 4.0
%%% International de Creative Commons.
%%% http://creativecommons.org/licenses/by-sa/4.0/

\begingroup
\calccentering{\unitlength}
\begin{adjustwidth*}{\unitlength}{-\unitlength}
  \setlength{\parindent}{0pt}
  \setlength{\parskip}{\baselineskip}

  {\textcopyright} {\year} Vincent Goulet \\

  \includegraphics[height=7mm,keepaspectratio=true]{by-sa}\\%
Cette création est mise à disposition selon le contrat
\href{http://creativecommons.org/licenses/by-sa/2.5/ca/}{%
  Paternité-Partage à l'identique 2.5 Canada} de Creative Commons. En
vertu de ce contrat, vous êtes libre de:
\begin{itemize}
\item \textbf{partager} —-- reproduire, distribuer et communiquer
  l'{\oe}uvre;
\item \textbf{remixer} —-- adapter l'{\oe}uvre;
\item utiliser cette {\oe}uvre à des fins commerciales.
\end{itemize}
Selon les conditions suivantes:

\begin{tabularx}{\linewidth}{@{}lX@{}}
  \raisebox{-9mm}[0mm][13mm]{%
    \includegraphics[height=11mm,keepaspectratio=true]{by}} &
  \textbf{Attribution} —-- Vous devez attribuer l'{\oe}uvre de la
  manière indiquée par l'auteur de l'{\oe}uvre ou le titulaire des
  droits (mais pas d'une manière qui suggérerait qu'ils vous
  soutiennent ou
  approuvent votre utilisation de l'{\oe}uvre). \\
  \raisebox{-9mm}{\includegraphics[height=11mm,keepaspectratio=true]{sa}}
  & \textbf{Partage à l'identique} --— Si vous modifiez, transformez
  ou adaptez cette {\oe}uvre, vous n'avez le droit de distribuer votre
  création que sous une licence identique ou similaire à celle-ci.
\end{tabularx}


  \textbf{Code source} \\
  \viewsource{\ghurl}

  \textbf{Couverture} \\
  Le reptile en couverture est un caméléon tapis (\emph{Furcifer
    lateralis}) originaire de Madagascar. Adulte, sa taille atteint
  les 25~cm, queue comprise.

  Crédit photo: Michabln Schwarz; \link{http://fc-foto.de/2077174}{}
\end{adjustwidth*}
\endgroup

%%% Local Variables:
%%% mode: latex
%%% TeX-master: "methodes-numeriques-en-actuariat_algebre-lineaire.tex"
%%% coding: utf-8
%%% End:
