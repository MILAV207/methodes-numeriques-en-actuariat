\documentclass[letterpaper,11pt]{memoir}
  \usepackage[utf8]{inputenc}
  \usepackage{fourier,microtype}       % texte en Utopia
  \usepackage[scaled=0.82]{beramono}   % code en Bera
  \renewcommand{\sfdefault}{Myriad-LF} % titres en MyriadPro
  \usepackage{natbib,url}
  \usepackage[english,francais]{babel}
  \usepackage[autolanguage]{numprint}
  \usepackage{vgmath,actu,amsmath,amsthm,amsfonts,icomma}
  \usepackage[noae]{Sweave}
  \usepackage{paralist,enumitem}
  \usepackage{textpos}
  \usepackage{graphicx,color}
  \usepackage{MnSymbol,textcomp}       % symboles spéciaux
  \usepackage{listingsutf8,answers}
  \usepackage{pdfpages}                % couvertures
  \usepackage{xr}                      % références aux chap. 3 et 6

  %%% Références externes
  \externaldocument{operateurs}
  \externaldocument{avance}

  %%%  Couleurs
  \definecolor{comments}{rgb}{0.7,0,0}
  \definecolor{darkred}{rgb}{0.8,0,0}
  \definecolor{shadecolor}{gray}{0}
  \definecolor{link}{rgb}{0,0,0.3}

  %%% Hyperliens
  \usepackage{hyperref}
  \hypersetup{colorlinks,allcolors=link}

  %%% ============
  %%%  Page titre
  %%% ============
  \title{%
    \fontseries{b}\fontsize{42}{33}\selectfont ACT 2002 \\
    \fontseries{m}\fontsize{32}{33}\selectfont Méthodes numériques \\
                                               en actuariat \\[20mm]
    \fontseries{b}\fontsize{36}{36}\selectfont Partie III \\
    \fontseries{m}\fontsize{32}{36}\selectfont Analyse numérique}
  \author{%
    \fontseries{b}\fontsize{16}{20}\selectfont Vincent Goulet \\
    \fontseries{m}\fontsize{14}{18}\selectfont Professeur titulaire \textbar\
                                               École d'actuariat \textbar\
                                               Université Laval}
  \date{%
    \fontseries{m}\fontsize{14}{18}\selectfont Notes de cours \textbar\
                                               Exercices}

  %%% ===================
  %%%  STYLE DU DOCUMENT
  %%% ===================

  %% Titres des chapitres
  \chapterstyle{hangnum}
  \renewcommand{\chaptitlefont}{\normalfont\Huge\sffamily\bfseries\raggedright}

  %% Marges, entêtes et pieds de page
  \setlength{\marginparsep}{7mm}
  \setlength{\marginparwidth}{13mm}
  \setlength{\headwidth}{\textwidth}
  \addtolength{\headwidth}{\marginparsep}
  \addtolength{\headwidth}{\marginparwidth}

  %% Titres des sections et sous-sections
  \setsecheadstyle{\normalfont\Large\sffamily\bfseries\raggedright}
  \setsubsecheadstyle{\normalfont\large\sffamily\bfseries\raggedright}
  \maxsecnumdepth{subsection}
  \setsecnumdepth{subsection}

  %% Listes. Paramétrage avec enumitem.
  \setenumerate{leftmargin=*,align=left}
  \setenumerate[2]{label=\alph*)}
  \setenumerate[3]{label=\roman*),align=right}
  \setitemize{leftmargin=*,align=left}

  %% Noms de fonctions, code, etc.
  \newcommand{\code}[1]{\texttt{#1}}
  \newcommand{\pkg}[1]{\textbf{#1}}

  %% Environnements d'exemples et al.
  \theoremstyle{plain}
  \newtheorem{algorithme}{Algorithme}[chapter]
  \newtheorem{thm}{Théorème}[chapter]

  \theoremstyle{definition}
  \newtheorem{exemple}{Exemple}[chapter]
  \newtheorem{definition}{Définition}[chapter]
  \newtheorem*{astuce}{Astuce}

  \theoremstyle{remark}
  \newtheorem*{remarque}{Remarque}
  \newtheorem*{remarques}{Remarques}
  \newenvironment{rem}{\begin{remarque} \mbox{}}{\end{remarque}}
  \newenvironment{rems}{\begin{remarques} \mbox{}}{\end{remarques}}

  %% Options de babel
  \frenchbsetup{CompactItemize=false,%
    ThinSpaceInFrenchNumbers=true,
    ItemLabeli=$\filledtriangleright$,
    ItemLabelii=\textendash}
  \addto\captionsfrench{\def\figurename{{\scshape Fig.}}}
  \addto\captionsfrench{\def\tablename{{\scshape Tab.}}}

  %%% =========================
  %%%  Nouveaux environnements
  %%% =========================

  %% Listes d'objectifs au début des chapitres
  \newenvironment{objectifs}{%
    \noindent
    \begin{framed}
      \vspace{-1.33\baselineskip}
      \begin{shaded}
        \noindent\sffamily\bfseries\textcolor{white}{Objectifs du chapitre}
      \end{shaded}
      \vspace{-0.6\baselineskip}
      \begin{itemize}[nolistsep]
        \small}
      {\end{itemize}
    \end{framed}}

  %% Environnements de Sweave
  \DefineVerbatimEnvironment{Sinput}{Verbatim}{xleftmargin=\parindent}
  \DefineVerbatimEnvironment{Soutput}{Verbatim}{xleftmargin=\parindent}
  \DefineVerbatimEnvironment{Scode}{Verbatim}{xleftmargin=\parindent}
  \fvset{listparameters={\setlength{\topsep}{0pt}}}
  \renewenvironment{Schunk}{\vspace{\topsep}}{\vspace{\topsep}}

  %% Exercices et réponses
  \Newassociation{sol}{solution}{solutions}
  \Newassociation{rep}{reponse}{reponses}
  \newcounter{exercice}[chapter]
  \newenvironment{exercice}{%
    \begin{list}{\bfseries \thechapter.\arabic{exercice}}{%
        \refstepcounter{exercice}
        \settowidth{\labelwidth}{\bfseries \thechapter.\arabic{exercice}}
        \setlength{\leftmargin}{\labelwidth}
        \addtolength{\leftmargin}{\labelsep}
        \setenumerate[1]{leftmargin=*,label=\alph*),widest=a}
        \setenumerate[2]{leftmargin=*,label=\roman*)}}
      \item}
    {\end{list}}
  \renewenvironment{reponse}[1]{%
    \begin{enumerate}[label=#1,font=\bfseries]
      \setenumerate[2]{leftmargin=*,labelsep=0em}
      \item}
    {\end{enumerate}}
  \renewcommand{\reponseparams}{{\thechapter.\theexercice}}
  \renewenvironment{solution}[1]{%
    \begin{enumerate}[label=#1,font=\bfseries]
      \setenumerate[2]{leftmargin=*,labelsep=0em}
      \item}
    {\end{enumerate}}
  \renewcommand{\solutionparams}{{\thechapter.\theexercice}}

  %%% Changements au fil de la lecture
  \newenvironment{gotoR}{%
    \begin{framed}%
      \noindent
      \begin{minipage}{0.07\linewidth}
        \raisebox{-0.5em}[0em][0em]{\LARGE\color{darkred}\noway}
      \end{minipage}
      \begin{minipage}[t]{0.88\linewidth}}
      {\end{minipage}%
    \end{framed}}

  %%% Remarques importantes
  \newenvironment{important}{%
    \begin{framed}%
      \noindent
      \begin{minipage}{0.1\linewidth}
        \raisebox{-0.5em}[0em][0em]{\LARGE\danger}
      \end{minipage}
      \begin{minipage}[t]{0.85\linewidth}}
      {\end{minipage}%
    \end{framed}}

  %%% =============================================
  %%%  Paramètres pour les sections de code source
  %%% =============================================
  \lstloadlanguages{R}
  \lstset{language=R,
    extendedchars=true,
    inputencoding=utf8/latin1,
    basicstyle=\small\ttfamily,
    commentstyle=\color{comments}\slshape,
    keywordstyle=\mdseries,
    showstringspaces=false}

  %%% =====================
  %%%  Nouvelles commandes
  %%% =====================
  \newcommand{\R}{\mathbb{R}}
  \newcommand{\Fonction}[1]{\code{#1}}
  \newcommand{\fonction}[1]{\code{#1}}
  \newcommand{\ieee}[3]{\fbox{#1}\hspace{2pt}\fbox{#2}\hspace{2pt}\fbox{#3}}
  \newcommand{\fl}{\mathrm{fl}}
  \newcommand{\texthl}[1]{\textbf{\color{darkred}{#1}}}

  %%% Support pour sous-tableaux et sous-figures
  \newsubfloat{table}
  \newsubfloat{figure}

  %%% Aide pour la césure
  \hyphenation{hexa-dé-ci-mal}

  %%% Style de la bibliographie
  \bibliographystyle{francais}

  %%% Numérotation des chapitres
  \setcounter{chapter}{9}

%  \includeonly{pagegarde,notices}

\begin{document}

\frontmatter

\pagestyle{empty}

%% Page couverture avant. Il faut modifier la largeur des graphiques
%% puisque Sweave la règle à 0.8\textwidth.
\setkeys{Gin}{width=\paperwidth}
\includepdf[pages=1]{couvertures-partie_3}
\setkeys{Gin}{width=0.8\textwidth}
\cleardoublepage

\begin{adjustwidth*}{-12mm}{-72mm}
  \sffamily
  \raggedright
  \vspace*{-17mm}
  \thetitle \\
  \vspace*{20mm}
  \theparttitle \\
  \vspace*{32mm}
  \theauthor \\
  \vspace*{\fill}
  \thedate
\end{adjustwidth*}

%%% Local Variables:
%%% mode: latex
%%% TeX-master: "methodes_numeriques-partie_3"
%%% coding: utf-8
%%% End:

\clearpage

\begingroup
\calccentering{\unitlength}
\begin{adjustwidth*}{\unitlength}{-\unitlength}
  \setlength{\parindent}{0pt}
  \setlength{\parskip}{\baselineskip}

  {\textcopyright} {\year} Vincent Goulet \\

  \includegraphics[height=7mm,keepaspectratio=true]{by-sa}\\%
Cette création est mise à disposition selon le contrat
\href{http://creativecommons.org/licenses/by-sa/2.5/ca/}{%
  Paternité-Partage à l'identique 2.5 Canada} de Creative Commons. En
vertu de ce contrat, vous êtes libre de:
\begin{itemize}
\item \textbf{partager} —-- reproduire, distribuer et communiquer
  l'{\oe}uvre;
\item \textbf{remixer} —-- adapter l'{\oe}uvre;
\item utiliser cette {\oe}uvre à des fins commerciales.
\end{itemize}
Selon les conditions suivantes:

\begin{tabularx}{\linewidth}{@{}lX@{}}
  \raisebox{-9mm}[0mm][13mm]{%
    \includegraphics[height=11mm,keepaspectratio=true]{by}} &
  \textbf{Attribution} —-- Vous devez attribuer l'{\oe}uvre de la
  manière indiquée par l'auteur de l'{\oe}uvre ou le titulaire des
  droits (mais pas d'une manière qui suggérerait qu'ils vous
  soutiennent ou
  approuvent votre utilisation de l'{\oe}uvre). \\
  \raisebox{-9mm}{\includegraphics[height=11mm,keepaspectratio=true]{sa}}
  & \textbf{Partage à l'identique} --— Si vous modifiez, transformez
  ou adaptez cette {\oe}uvre, vous n'avez le droit de distribuer votre
  création que sous une licence identique ou similaire à celle-ci.
\end{tabularx}


  \textbf{Code source} \\
  \begin{tabularx}{1.0\linewidth}{@{}Xl@{}}
    Code informatique des sections d'exemples & \href{http://libre.act.ulaval.ca/fileadmin/Portail_libre/ACT-2002/Notes\%20de\%20cours/code-partie_1.zip}{\downloadbutton} \\
    \addlinespace[3pt]
    Sorties du code informatique & \href{http://libre.act.ulaval.ca/fileadmin/Portail_libre/ACT-2002/Notes\%20de\%20cours/code-partie_1-sorties.zip}{\downloadbutton} \\
    \addlinespace[3pt]
    Code source du document & \href{https://svn.fsg.ulaval.ca/svn-pub/vgoulet/documents/methodes_numeriques/}{\browsebutton}
  \end{tabularx}

  \textbf{Couverture} \\
  Le reptile en couverture est un caméléon panthère (\emph{Furcifer
    pardalis}) originaire de Madagascar. Il s'agit d'un des plus
  grands caméléons existants, la taille du mâle pouvant atteindre
  55~cm, queue comprise.

  Crédit photo: Maria Stenzel, National Geographic Society
\end{adjustwidth*}
\endgroup

%%% Local Variables:
%%% mode: latex
%%% TeX-master: "methodes_numeriques-partie_1"
%%% coding: utf-8
%%% End:

\clearpage

\pagestyle{companion}

\chapter*{Introduction}
\addcontentsline{toc}{chapter}{Introduction}
\markboth{Introduction}{Introduction}

Les ordinateurs ne savent pas compter ou, en fait, très peu. Ils ne
savent traiter que des 0 et des 1 et sont incapables de représenter
tous les nombres réels --- chose qu'un humain peut faire, du moins
conceptuellement. Cela signifie qu'à peu près tout calcul effectué
dans un ordinateur comporte une part d'erreur d'arrondi et de
troncature. Comme on ne souhaite généralement pas que cette erreur
devienne trop grande, il importe de connaître ses sources afin de la
diminuer le plus possible. C'est, entre autres choses, l'objet du
\autoref{chap:ordinateurs}.

Les procédures numériques pour résoudre des équations à une variable,
optimiser une fonction ou calculer une intégrale définie sont
aujourd'hui aisément accessibles dans une foule de logiciels à
connotation mathématique ou même dans une simple calculatrice. Or,
comment ces calculs sont-ils effectués, quels sont les algorithmes à
l'{\oe}uvre en arrière-scène? Le \autoref{chap:resolution} se
penche sur les méthodes de base de résolution d'équations à une
variable et le \autoref{chap:integration} sur celles
d'intégration numérique.

On présente également au \autoref{chap:resolution} les
principales fonctions d'optimisation disponibles dans Excel et dans R.

L'étude de ce document implique quelques allers-retours entre le texte
et les sections de code informatique présentes dans chaque chapitre.
Les sauts vers ces sections sont clairement indiqués dans le texte par
des mentions mises en évidence par le symbole {\ForwardToEnd}.

Chaque chapitre comporte des exercices. Les réponses de ceux-ci se
trouvent à la fin de chacun des chapitres et les solutions complètes,
en annexe.

Je tiens à souligner la précieuse collaboration de MM.~Mathieu
Boudreault, Sébastien Auclair et Louis-Philippe Pouliot lors de la
rédaction des exercices et des solutions. Je remercie également
Mmes~Marie-Pier Laliberté et Véronique Tardif pour l'infographie des
pages couvertures.

%%% Local Variables:
%%% mode: latex
%%% TeX-master: "methodes_numeriques-partie_3"
%%% End:

\cleartorecto
\tableofcontents*

\mainmatter

\include{arithmetique_ordinateurs}
\include{resolution_equations}
\include{integration_numerique}

\appendix
\chapter{Solutions des exercices}
\label{chap:solutions}
\markboth{Solutions des exercices}{Solutions des exercices}

\input{solutions-arithmetique_ordinateurs}
\input{solutions-resolution_equations}
\input{solutions-integration_numerique}

%%% Local Variables:
%%% mode: latex
%%% TeX-master: "methodes_numeriques-partie_3"
%%% coding: utf-8
%%% End:


\bibliography{r,math,stat,informatique,vg}

%%% Copyright (C) 2018 Vincent Goulet
%%%
%%% Ce fichier fait partie du projet
%%% «Méthodes numériques en actuariat avec R»
%%% http://github.com/vigou3/methodes-numeriques-en-actuariat
%%%
%%% Cette création est mise à disposition selon le contrat
%%% Attribution-Partage dans les mêmes conditions 4.0
%%% International de Creative Commons.
%%% http://creativecommons.org/licenses/by-sa/4.0/

\vspace*{\fill}

\begingroup
\calccentering{\unitlength}
\begin{adjustwidth*}{\unitlength}{-\unitlength}
  \begin{flushleft}
    \small %
    Ce document a été produit avec le système de mise en page
    {\XeLaTeX}. Le texte principal est en Lucida Bright~OT 11~points,
    les mathématiques en Lucida Bright Math~OT, le code informatique
    en Lucida Grande Mono~DK et les titres en Adobe Myriad~Pro. Des
    icônes proviennent de la police Font~Awesome. Les graphiques ont
    été réalisés avec R.
  \end{flushleft}
\end{adjustwidth*}
\endgroup
\vfill


\cleardoublepage
\cleartoverso

%% Page couverture arrière.
\setkeys{Gin}{width=\paperwidth}
\includepdf[pages=2]{couvertures-partie_3}

\end{document}

%%% Local Variables:
%%% mode: latex
%%% TeX-master: t
%%% End:
