\documentclass[letterpaper,10pt]{memoir}
  \usepackage[T1]{fontenc}
  \usepackage[utf8]{inputenc}
  \usepackage[english,francais]{babel}
  \usepackage{vgmath,vgsets,amsmath,actu,icomma,url,natbib}
  \usepackage{lucidabr,palatino,mathpazo}
  \usepackage[noae]{Sweave}
  \usepackage{graphicx,color,multicol}
  \usepackage[absolute]{textpos}
  \usepackage{answers}
  \usepackage[alwaysadjust]{paralist}

  %%% Page titre
  \title{\HUGE
    \fontseries{ub}\selectfont Exercices \\
    \fontseries{m}\selectfont  de \\
    \fontseries{ub}\selectfont méthodes numériques \\[0.5\baselineskip]
    \huge\fontseries{m}\selectfont Avec solutions}
  \author{\LARGE Vincent Goulet \\[3mm]
    \large École d'actuariat, Université Laval}
  \date{}
  \newcommand{\ISBN}{978-2-9809136-8-6}

  %%% Style des entêtes de chapitres
  \chapterstyle{hangnum}

  %%% Styles des entêtes et pieds de page
  \setlength{\marginparsep}{7mm}
  \setlength{\marginparwidth}{13mm}
  \setlength{\headwidth}{\textwidth}
  \addtolength{\headwidth}{\marginparsep}
  \addtolength{\headwidth}{\marginparwidth}

  %%% Style de la bibliographie
  \bibliographystyle{francais}

  %%% Légendes des figures en italique
  \captiontitlefont{\itshape}
  \captiondelim{ --- }

  %%% Associations entre les environnements et les fichiers
  \Newassociation{sol}{solution}{solutions}
  \Newassociation{rep}{reponse}{reponses}

  %%% Environnement pour les exercices
  \newcounter{exercice}[chapter]
  \newenvironment{exercice}{%
     \begin{list}{\bfseries \arabic{chapter}.\arabic{exercice}}{%
         \refstepcounter{exercice}
         \settowidth{\labelwidth}{\bfseries \arabic{chapter}.\arabic{exercice}}
         \setlength{\leftmargin}{\labelwidth}
         \addtolength{\leftmargin}{\labelsep}
         \setdefaultenum{a)}{i)}{}{}}\item}
     {\end{list}}

  %%% Environnement pour les réponses
  \renewenvironment{reponse}[1]{%
    \begin{list}{\bfseries #1}{%
        \settowidth{\labelwidth}{#1}
        \setlength{\leftmargin}{\labelwidth}
        \addtolength{\leftmargin}{\labelsep}
        \setdefaultenum{a)}{i)}{}{}}\item}
    {\end{list}}
  \renewcommand{\reponseparams}{{\thechapter.\theexercice}}

  %%% Environnement pour les solutions
  \renewenvironment{solution}[1]{%
    \begin{list}{\bfseries #1}{%
        \settowidth{\labelwidth}{#1}
        \setlength{\leftmargin}{\labelwidth}
        \addtolength{\leftmargin}{\labelsep}
        \setdefaultenum{a)}{i)}{}{}}\item}
    {\end{list}}
  \renewcommand{\solutionparams}{{\thechapter.\theexercice}}

  %%% Nouvelles commandes
  \newcommand{\ieee}[3]{%
    \fbox{#1}\hspace{2pt}\fbox{#2}\hspace{2pt}\fbox{#3}}
  \newcommand{\tr}{\mathrm{tr}}

  %%% Environnements pour le code S: police plus petite
  \RecustomVerbatimEnvironment{Sinput}{Verbatim}{fontshape=sl,fontsize=\small}
  \RecustomVerbatimEnvironment{Soutput}{Verbatim}{fontsize=\small}
  \RecustomVerbatimEnvironment{Scode}{Verbatim}{fontsize=\small}

  %%% Support pour sous-figures
  \newsubfloat{figure}

  %%% Nouvelle longueur utilisée dans les solutions d'algèbre
  %%% linéaire.
  \newlength{\ocolumnsep}

  %%% Petit coup de pouce pour la césure
  \hyphenation{con-gruen-tiel}
  \hyphenation{con-gruen-tiels}

%\includeonly{pagetitre}

\begin{document}

\shortcites{Rintro}

\frontmatter

\pagestyle{empty}
%%% Police de caractère pour la page titre
\renewcommand{\sfdefault}{hls}

%%% Marge de gauche (1/3 de la page)
\newlength{\gauche}
\addtolength{\gauche}{72mm}
\addtolength{\gauche}{-\spinemargin}

%%% Épaisseur de la bande sur la page titre
\newlength{\ruleheight}
\setlength{\ruleheight}{7.75mm}

%%% Définition de la bande
\definecolor{gray}{gray}{0.4}
\textblockorigin{0mm}{279mm}
\newcommand{\banderecto}{%
  \begin{textblock*}{71.5mm}[0,1](0mm,-46.5mm)
    \textblockcolor{black} \rule{0mm}{\ruleheight}
  \end{textblock*}
  \begin{textblock*}{144mm}[0,1](72mm,-46.5mm)
    \textblockcolor{gray} \rule{0mm}{\ruleheight}
  \end{textblock*}}
\newcommand{\bandeverso}{%
  \begin{textblock*}{144mm}[0,1](0mm,-46.5mm)
    \textblockcolor{gray} \rule{0mm}{\ruleheight}
  \end{textblock*}
  \begin{textblock*}{71.5mm}[0,1](144.5mm,-46.5mm)
    \textblockcolor{black} \rule{0mm}{\ruleheight}
  \end{textblock*}}

%%% Titre
\begin{adjustwidth*}{\gauche}{-15mm}
  \sffamily\fontseries{ub}\selectfont
  \raggedright
  \vspace*{2cm}
  \thetitle
\end{adjustwidth*}

%%% Affichage de la bande
\banderecto
\cleardoublepage

%%% Page de garde
\begin{adjustwidth*}{\gauche}{-15mm}
  \sffamily\fontseries{ub}\selectfont
  \raggedright
  \vspace*{2cm}
  \thetitle \\
  \bfseries
  \vspace*{3cm}
  \theauthor \\
  \vspace*{\fill}
  \thedate
\end{adjustwidth*}
\clearpage

%%% Page de notices
\begingroup
\calccentering{\unitlength}
\begin{adjustwidth*}{\unitlength}{-\unitlength}
  \small
  \setlength{\parindent}{0pt}
  \setlength{\parskip}{\baselineskip}

  \textcopyright{} 2008 Vincent Goulet \\

  Cette création est mise à disposition selon le contrat
  Paternité-Partage des conditions initiales à l'identique 2.5 Canada
  disponible en ligne
  \url{http://creativecommons.org/licenses/by-sa/2.5/ca/} ou par
  courrier postal à Creative Commons, 171 Second Street, Suite 300,
  San Francisco, California 94105, USA.

  \begin{center}
    \begin{tabular}{ll}
      Version 1.0: janvier 2008 &
    \end{tabular}
  \end{center}

  Le code source {\LaTeX} de ce document est disponible à l'adresse
  \begin{adjustwidth}{\parindent}{\parindent}
    \url{http://vgoulet.act.ulaval.ca/methodes_numeriques/}
  \end{adjustwidth}
  ou en communiquant directement avec l'auteur.

  \vspace{1cm}

  ISBN \ISBN \\
  Dépôt légal -- Bibliothèque et Archives nationales du Québec, 2008 \\
  Dépôt légal -- Bibliothèque et Archives Canada, 2008
\end{adjustwidth*}
\endgroup

%%% Retour à la police normale
\renewcommand{\sfdefault}{phv}


%%% Local Variables:
%%% mode: latex
%%% TeX-master: "exercices_methodes_numeriques"
%%% End:


\pagestyle{companion}

\chapter*{Introduction}
\addcontentsline{toc}{chapter}{Introduction}
\markboth{Introduction}{Introduction}


Ce document est une collection des exercices distribués par l'auteur
dans ses cours de Méthodes numériques en actuariat entre 2005 et 2007,
cours donnés à l'École d'actuariat de l'Université Laval. Certains
exercices sont le fruit de l'imagination de l'auteur, alors que
plusieurs autres sont des adaptations d'exercices tirés des ouvrages
cités dans la bibliographie.

C'est d'ailleurs afin de ne pas usurper de droits d'auteur que ce
document est publié selon les termes du contrat Paternité-Partage des
conditions initiales 2.5 Canada de Creative Commons. Il s'agit donc
d'un document «libre» que quiconque peut réutiliser et modifier à sa
guise, à condition que le nouveau document soit publié avec le même
contrat.

Le cours de Méthodes numériques est séparé en quatre parties plus ou
moins étanches les unes aux autres:
\begin{enumerate}[I.]
\item Introduction à la programmation en S;
\item Simulation stochastique;
\item Analyse numérique;
\item Algèbre linéaire.
\end{enumerate}
Tant la théorie que les exercices de la première partie se trouvent
dans \cite{Goulet_intro_S}. Ce document contient donc les exercices
relatifs aux parties II--IV. Nous invitons le lecteur à consulter,
entre autres, \cite{Ripley_87}, \cite{Gentle_98}, \cite{BurdenFaires}
et \cite{Anton8e} pour d'excellents exposés sur les sujets pré-cités.

Les réponses des exercices se trouvent à la fin de chacun des
chapitres et les solutions complètes en annexe du document.

Nous remercions d'avance les lecteurs qui voudront bien nous faire
part de toute erreur ou omission dans les exercices ou leurs
solutions.

Enfin, nous tenons à remercier MM.~Mathieu Boudreault, Sébastien
Auclair et Louis-Philippe Pouliot pour leur précieuse collaboration
lors de la rédaction des exercices et des solutions.

\begin{flushright}
  \itshape
  Vincent Goulet \url{<vincent.goulet@act.ulaval.ca>} \\
  Québec, décembre 2007
\end{flushright}

%%% Local Variables:
%%% mode: latex
%%% TeX-master: "exercices_methodes_statistiques"
%%% End:


\cleardoublepage
\tableofcontents*

\mainmatter

\stepcounter{part}
\stepcounter{chapter}

\part{Simulation stochastique}
\label{part:simulation}

\include{generateurs}
\include{simulation_va}

\part{Analyse numérique}
\label{part:analysenumerique}

\include{arithmetique_ordinateurs}
\include{resolution_equations}

\part{Algèbre linéaire}
\label{part:algebrelineaire}

\include{revision_algebre_lineaire}
\include{valeurs_propres}
\chapter{Méthodes de résolution de systèmes d'équations linéaires}
\label{chap:decomposition}

L'utilisation des matrices de loin la plus fréquente en actuariat
consiste à résoudre des systèmes d'équations linéaires du type
\begin{displaymath}
  \mat{A} \mat{x} = \mat{b}.
\end{displaymath}
Nous avons jusqu'à maintenant étudié diverses façons --- toutes
équivalentes d'un point de vue mathématique --- d'obtenir la solution
\begin{displaymath}
  \mat{x} = \mat{A}^{-1} \mat{b}.
\end{displaymath}
On peut maintenant se demander laquelle est la plus efficace d'un
point de vue informatique, surtout lorsque le système compte un grand
nombre d'équations.


\section{Comparaison du nombre d'opérations}
\label{sec:decomposition:nombre_operations}

Les principales méthodes de résolution d'un système d'équations
linéaires sont:
\begin{enumerate}
\item l'élimination gaussienne avec substitution successive;
\item l'élimination de Gauss--Jordan;
\item le calcul de $\mat{A}^{-1}$, puis de $\mat{x} = \mat{A}^{-1}
  \mat{b}$.
\end{enumerate}
Le calcul de $\mat{A}^{-1}$ peut quant à lui s'effectuer par
transformation de $[\mat{A}|\mat{I}]$ en $[\mat{I}|\mat{A}^{-1}]$, ou
par la méthode des cofacteurs.

L'élément décisif dans la comparaison des temps de calcul de ces
diverses méthodes de calcul est le nombre d'opérations arithmétiques
requis. Le tableau \ref{tab:decomposition:nb_oper} présente l'ordre de
grandeur du nombre de multiplications et divisions (et non le nombre
exact) nécessaire pour obtenir une réponse avec chacune des méthodes
ci-dessus. On se concentre sur les multiplications et divisions,
sachant que ces opérations coûtent plus cher en temps de calcul que
les additions et soustractions.
\begin{table}
  \centering
  \begin{tabular}{lc}
    \toprule
    Méthode & Nombre d'opérations \\
    \midrule
    Élimination gaussienne & $n^3/3$ \\
    Élimination de Gauss--Jordan & $n^3/3$ \\
    Transformation de $[\mat{A}|\mat{I}]$ en $[\mat{I}|\mat{A}^{-1}]$ &
    $n^3$ \\
    Calcul de $\mat{x} = \mat{A}^{-1} \mat{b}$ & $n^3$ \\
    \bottomrule
  \end{tabular}
  \caption{Nombre approximatif de multiplications et divisions pour
    résoudre le système d'équations à $n$ équations et $n$ inconnues
    $\mat{A} \mat{x} = \mat{b}$}
  \label{tab:decomposition:nb_oper}
\end{table}

On constate que l'élimination gaussienne avec substitution successive
et l'élimination de Gauss--Jordan sont les méthodes les plus rapides,
leur avantage augmentant rapidement avec la taille du système
d'équations.


\section{Décomposition $LU$}
\label{sec:decomposition:decomposition}

Le principal inconvénient des méthodes d'élimination réside dans le
fait qu'il faut connaître le vecteur des coefficients $\mat{b}$ au
moment d'effectuer les calculs. Si l'on souhaite résoudre le système
$\mat{Ax} = \mat{b}$ pour un nouveau vecteur $\mat{b}$, il faut
répéter la procédure depuis le début.

Les principales routines de résolution de systèmes d'équations
linéaires disponibles dans les divers outils informatiques\footnote{%
  Plusieurs utilisent les librairies LINPACK ou LAPACK.}  (S-Plus,
\textsf{R}, Maple, Matlab, Mathematica, etc.) reposent donc plutôt sur
la technique de la décomposition $LU$, une variante de l'élimination
gaussienne ne nécessitant pas de connaître d'avance le vecteur
$\mat{b}$.

L'idée est très simple: si la matrice $\mat{A}$ peut être factorisée
en un produit de matrices $n \times n$
\begin{displaymath}
  \mat{A} = \mat{L} \mat{U},
\end{displaymath}
où $\mat{L}$ est une matrice triangulaire inférieure et $\mat{U}$ une
matrice triangulaire supérieure, alors on obtient le système
d'équations linéaires
\begin{equation}
  \mat{L} \mat{U} \mat{x} = \mat{b}.
  \label{eq:decomposition:LUx}
\end{equation}
Or, en posant
\begin{equation}
  \mat{U} \mat{x} = \mat{y}
  \label{eq:decomposition:Ux}
\end{equation}
on peut réécrire \eqref{eq:decomposition:LUx} comme
\begin{equation}
  \mat{L} \mat{y} = \mat{b}.
\end{equation}
La solution $\mat{y}$ de ce dernier système d'équations est simple à
obtenir par substitutions successives. De même, une fois $\mat{y}$
connu, le vecteur $\mat{x}$ est obtenu en résolvant
\eqref{eq:decomposition:Ux}, toujours par simples substitutions.

\begin{exemple}
  Soit le système d'équations linéaires
  \begin{displaymath}
    \begin{bmatrix}
       2 &  6 & 2 \\
      -3 & -8 & 0 \\
       4 &  9 & 2
    \end{bmatrix}
    \begin{bmatrix}
      x_1 \\ x_2 \\ x_3
    \end{bmatrix}
    =
    \begin{bmatrix}
      2 \\ 2 \\ 3
    \end{bmatrix}.
  \end{displaymath}
  On peut démontrer que
  \begin{displaymath}
    \mat{A} =
    \begin{bmatrix}
       2 &  6 & 2 \\
      -3 & -8 & 0 \\
       4 &  9 & 2
    \end{bmatrix}
    =
    \begin{bmatrix}
       2 &  0 & 0 \\
      -3 &  1 & 0 \\
       4 & -3 & 7
    \end{bmatrix}
    \begin{bmatrix}
      1 & 3 & 1 \\
      0 & 1 & 3 \\
      0 & 0 & 1
    \end{bmatrix},
  \end{displaymath}
  soit $\mat{A} = \mat{L} \mat{U}$ avec
  \begin{align*}
    \mat{L} &=
    \begin{bmatrix}
       2 &  0 & 0 \\
      -3 &  1 & 0 \\
       4 & -3 & 7
    \end{bmatrix} \\
    \intertext{et}
    \mat{U} &=
    \begin{bmatrix}
      1 & 3 & 1 \\
      0 & 1 & 3 \\
      0 & 0 & 1
    \end{bmatrix}.
  \end{align*}
  On a donc $\mat{A} \mat{x} = \mat{L} \mat{U} \mat{x} =
  \mat{b}$. En posant $\mat{U} \mat{x} = \mat{y}$, on réécrit le
  système d'équations sous la forme $\mat{L} \mat{y} = \mat{b}$, soit
  \begin{displaymath}
    \begin{bmatrix}
       2 &  0 & 0 \\
      -3 &  1 & 0 \\
       4 & -3 & 7
    \end{bmatrix}
    \begin{bmatrix}
      y_1 \\ y_2 \\ y_3
    \end{bmatrix}
    =
    \begin{bmatrix}
      2 \\ 2 \\ 3
    \end{bmatrix}.
  \end{displaymath}
  Par substitution successive, on trouve
  \begin{align*}
    y_1 &= 1 \\
    y_2 &= 2 + 3 y_1 = 5 \\
    y_3 &= \frac{3 - 4 y_1 + 3 y_2}{7} = 2.
  \end{align*}
  Pour trouver la solution du système d'équations original, il suffit
  maintenant de résoudre $\mat{U} \mat{x} = \mat{y}$, soit
  \begin{displaymath}
    \begin{bmatrix}
      1 & 3 & 1 \\
      0 & 1 & 3 \\
      0 & 0 & 1
    \end{bmatrix}
    \begin{bmatrix}
      x_1 \\ x_2 \\ x_3
    \end{bmatrix}
    =
    \begin{bmatrix}
      1 \\ 5 \\ 2
    \end{bmatrix}.
  \end{displaymath}
  On obtient alors
  \begin{align*}
    x_3 &= 2 \\
    x_2 &= 5 - 3 x_3 = -1 \\
    x_1 &= 1 - 3 x_2 - x_3 = 2.
  \end{align*}
  \qed
\end{exemple}

L'essentiel des calculs se retrouve dans la factorisation de la
matrice $\mat{A}$ en un produit de matrices triangulaires. Pour
justifier la technique, supposons que l'on réduit la matrice $\mat{A}$
sous forme échelonnée par une série d'opérations élémentaires sur les
lignes. On peut donc trouver des matrices élémentaires $\mat{E}_1,
\dots, \mat{E}_k$ tel que
\begin{displaymath}
  \mat{E}_k \cdots \mat{E}_1 \mat{A} = \mat{U}.
\end{displaymath}
Par le théorème \ref{thm:revision:matrices_elementaires}, l'inverse
d'une matrice élémentaire existe et est également une matrice
élémentaire, d'où
\begin{displaymath}
  \mat{A} = \mat{E}_1^{-1} \cdots \mat{E}_k^{-1} \mat{U}
\end{displaymath}
et donc
\begin{displaymath}
  \mat{L} = \mat{E}_1^{-1} \cdots \mat{E}_k^{-1}.
\end{displaymath}
Cette dernière matrice est triangulaire inférieure \emph{à condition
  de ne pas échanger des lignes} lors de la réduction de $\mat{A}$
vers $\mat{U}$.  Il existe un algorithme simple pour construire la
matrice $\mat{L}$ sans devoir effectuer le produit des matrices
élémentaires inverses; consulter \citet[section 9.9]{Anton:linear:8e:2000}.

Le nombre d'opérations de la décomposition $LU$ est du même ordre que
les méthodes d'élimination. Par contre, on remarquera que la
factorisation est tout à fait indépendante du vecteur $\mat{b}$. Une
fois la factorisation connue, on peut donc résoudre plusieurs systèmes
d'équations différents utilisant tous la même matrice de coefficients
$\mat{A}$ sans devoir répéter une grande partie des calculs.

%%% Local Variables:
%%% mode: latex
%%% TeX-master: "methodes_numeriques"
%%% End:


\appendix
%%% Copyright (C) 2018 Vincent Goulet
%%%
%%% Ce fichier fait partie du projet «Méthodes numériques en actuariat»
%%% http://github.com/vigou3/methodes-numeriques-en-actuariat
%%%
%%% Cette création est mise à disposition selon le contrat
%%% Attribution-Partage dans les mêmes conditions 4.0
%%% International de Creative Commons.
%%% http://creativecommons.org/licenses/by-sa/4.0/

\chapter{Solutions des exercices}
\label{chap:solutions}
\markboth{Solutions des exercices}{Solutions des exercices}

\begingroup

%% Environnement Schunk simplifié pour l'affichage des réponses
\renewenvironment{Schunk}{%
  \setlength{\topsep}{0pt}
  \colorlet{shadecolor}{codebg}
  \begin{snugshade*}}%
  {\end{snugshade*}}

\input{solutions-notions_fondamentales}
\input{solutions-valeurs_propres}
\input{solutions-decomposition_lu}

\endgroup

%%% Local Variables:
%%% mode: latex
%%% TeX-engine: xetex
%%% TeX-master: "methodes-numeriques-en-actuariat_algebre-lineaire.tex"
%%% coding: utf-8
%%% End:


\nocite{Monahan_01,Rubinstein_81}
\bibliography{stat,math,vg}

\cleardoublepage
\cleartoverso

\pagestyle{empty}
\renewcommand{\ttdefault}{hlst}

\bandeverso
\begin{textblock*}{71mm}(145mm, -40mm)
  \large\ttfamily\raggedright
  \textblockcolor{}
  ISBN \\ \ISBN
\end{textblock*}

\end{document}

%%% Local Variables:
%%% mode: latex
%%% TeX-master: t
%%% End:
