%%% Page couverture avant. Il faut modifier la largeur des graphiques
%%% puisque Sweave la règle à 0.8\textwidth.
\setkeys{Gin}{width=\paperwidth}
\includepdf[pages=1]{couvertures-partie_1}
\setkeys{Gin}{width=0.8\textwidth}

\cleardoublepage

%%% Page de garde
\begin{adjustwidth*}{13.1mm}{-72mm}
  \sffamily
  \raggedright
  \vspace*{5mm}
  \thetitle \\
  \vspace*{3cm}
  \theauthor \\
  \vspace*{\fill}
  \thedate
\end{adjustwidth*}
\clearpage

%%% Page de notices
\begingroup
\calccentering{\unitlength}
\begin{adjustwidth*}{\unitlength}{-\unitlength}
  \small
  \setlength{\parindent}{0pt}
  \setlength{\parskip}{\baselineskip}

  {\textcopyright} 2012 Vincent Goulet \\

  \includegraphics[height=7mm,keepaspectratio=true]{by-sa}\\%
Cette création est mise à disposition selon le contrat
\href{http://creativecommons.org/licenses/by-sa/2.5/ca/}{%
  Paternité-Partage à l'identique 2.5 Canada} de Creative Commons. En
vertu de ce contrat, vous êtes libre de:
\begin{itemize}
\item \textbf{partager} —-- reproduire, distribuer et communiquer
  l'{\oe}uvre;
\item \textbf{remixer} —-- adapter l'{\oe}uvre;
\item utiliser cette {\oe}uvre à des fins commerciales.
\end{itemize}
Selon les conditions suivantes:

\begin{tabularx}{\linewidth}{@{}lX@{}}
  \raisebox{-9mm}[0mm][13mm]{%
    \includegraphics[height=11mm,keepaspectratio=true]{by}} &
  \textbf{Attribution} —-- Vous devez attribuer l'{\oe}uvre de la
  manière indiquée par l'auteur de l'{\oe}uvre ou le titulaire des
  droits (mais pas d'une manière qui suggérerait qu'ils vous
  soutiennent ou
  approuvent votre utilisation de l'{\oe}uvre). \\
  \raisebox{-9mm}{\includegraphics[height=11mm,keepaspectratio=true]{sa}}
  & \textbf{Partage à l'identique} --— Si vous modifiez, transformez
  ou adaptez cette {\oe}uvre, vous n'avez le droit de distribuer votre
  création que sous une licence identique ou similaire à celle-ci.
\end{tabularx}


  \textbf{Code source} \\
  Le code source {\LaTeX} de ce document est disponible à l'adresse
    \url{https://svn.fsg.ulaval.ca/svn-pub/vgoulet/documents/methodes_numeriques/}
  ou en communiquant directement avec l'auteur.

  \textbf{Couverture} \\
  Le reptile en couverture est un caméléon panthère (\emph{Furcifer
    pardalis}) originaire de Madagascar. Il s'agit d'un des plus
  grands caméléons existants, la taille du mâle pouvant atteindre
  55~cm, queue comprise.

  La photo est tirée du site de National Geographic.
\end{adjustwidth*}
\endgroup

\clearpage

%%% Local Variables:
%%% mode: latex
%%% TeX-master: "methodes_numeriques-partie_1"
%%% coding: utf-8
%%% End:
