\documentclass[letterpaper,11pt]{memoir}
  \usepackage[utf8]{inputenc}
  \usepackage{fourier,microtype}       % texte en Utopia
  \usepackage[scaled=0.82]{beramono}   % code en Bera
  \renewcommand{\sfdefault}{Myriad-LF} % titres en MyriadPro
  \usepackage{natbib,url}
  \usepackage[english,francais]{babel}
  \usepackage[autolanguage]{numprint}
  \usepackage{vgmath,actu,amsmath,amsthm,icomma}
  \usepackage[noae]{Sweave}
  \usepackage{paralist}
  \usepackage[shortlabels]{enumitem}
  \usepackage{textpos}
  \usepackage{graphicx,color}
  \usepackage{MnSymbol,applekeys}      % symboles spéciaux
  \usepackage{expdlist}
  \usepackage{listingsutf8,answers}
  \usepackage{pdfpages}                % couvertures

  %%%  Couleurs
  \definecolor{comments}{rgb}{0.7,0,0}
  \definecolor{shadecolor}{gray}{0}
  \definecolor{link}{rgb}{0,0,0.3}

  %%% Hyperliens
  \usepackage{hyperref}
  \hypersetup{colorlinks,allcolors=link}

  %%% ============
  %%%  Page titre
  %%% ============
  \title{%
    \fontseries{b}\fontsize{42}{33}\selectfont ACT 2002 \\
    \fontseries{m}\fontsize{32}{33}\selectfont Méthodes numériques \\
                                               en actuariat \\[20mm]
    \fontseries{b}\fontsize{36}{36}\selectfont Partie I \\
    \fontseries{m}\fontsize{32}{36}\selectfont Programmation en R}
  \author{%
    \fontseries{b}\fontsize{16}{20}\selectfont Vincent Goulet \\
    \fontseries{m}\fontsize{14}{18}\selectfont Professeur titulaire \textbar\
                                               École d'actuariat \textbar\
                                               Université Laval}
  \date{%
    \fontseries{m}\fontsize{14}{18}\selectfont Notes de cours \textbar\
                                               Exercices \textemdash\
                                               édition 2013}

  %%% ===================
  %%%  STYLE DU DOCUMENT
  %%% ===================

  %% Titres des chapitres
  \chapterstyle{hangnum}
  \renewcommand{\chaptitlefont}{\normalfont\Huge\sffamily\bfseries\raggedright}

  %% Marges, entêtes et pieds de page
  \setlength{\marginparsep}{7mm}
  \setlength{\marginparwidth}{13mm}
  \setlength{\headwidth}{\textwidth}
  \addtolength{\headwidth}{\marginparsep}
  \addtolength{\headwidth}{\marginparwidth}

  %% Titres des sections et sous-sections
  \setsecheadstyle{\normalfont\Large\sffamily\bfseries\raggedright}
  \setsubsecheadstyle{\normalfont\large\sffamily\bfseries\raggedright}
  \maxsecnumdepth{subsection}
  \setsecnumdepth{subsection}

  %% Listes. Paramétrage avec enumitem.
  \setenumerate{leftmargin=*,align=left}
  \setenumerate[2]{label=\alph*)}
  \setenumerate[3]{label=\roman*),align=right}
  \setitemize{leftmargin=*,align=left}

  %% Noms de fonctions, code, etc.
  \newcommand{\code}[1]{\texttt{#1}}

  %% Environnements d'exemples et al.
  \theoremstyle{definition}
  \newtheorem*{astuce}{Astuce}

  %% Options de babel
  \frenchbsetup{CompactItemize=false,%
    ThinSpaceInFrenchNumbers=true,
    ItemLabeli=$\filledtriangleright$,
    ItemLabelii=\textendash}
  \addto\captionsfrench{\def\figurename{{\scshape Fig.}}}
  \addto\captionsfrench{\def\tablename{{\scshape Tab.}}}

  %%% =========================
  %%%  Nouveaux environnements
  %%% =========================

  %% Listes d'objectifs au début des chapitres
  \newenvironment{objectifs}{%
    \noindent
    \begin{framed}
      \vspace{-1.33\baselineskip}
      \begin{shaded}
        \noindent\sffamily\bfseries\textcolor{white}{Objectifs du chapitre}
      \end{shaded}
      \vspace{-0.6\baselineskip}
      \begin{compactitem}
        \small}
      {\end{compactitem}
    \end{framed}}

  %% Environnements de Sweave
  \DefineVerbatimEnvironment{Sinput}{Verbatim}{xleftmargin=\parindent}
  \DefineVerbatimEnvironment{Soutput}{Verbatim}{xleftmargin=\parindent}
  \DefineVerbatimEnvironment{Scode}{Verbatim}{xleftmargin=\parindent}
  \fvset{listparameters={\setlength{\topsep}{0pt}}}
  \renewenvironment{Schunk}{\vspace{\topsep}}{\vspace{\topsep}}

  %% Exercices et réponses
  \Newassociation{rep}{reponse}{reponses}
  \newcounter{exercice}[chapter]
  \newenvironment{exercice}{%
    \begin{list}{\bfseries \thechapter.\arabic{exercice}}{%
        \refstepcounter{exercice}
        \settowidth{\labelwidth}{\bfseries \thechapter.\arabic{exercice}}
        \setlength{\leftmargin}{\labelwidth}
        \addtolength{\leftmargin}{\labelsep}
        \setenumerate[1]{leftmargin=*,label=\alph*),widest=a}
        \setenumerate[2]{leftmargin=*,label=\roman*)}}
      \item}
    {\end{list}}
  \renewenvironment{reponse}[1]{%
    \begin{enumerate}[label=#1,font=\bfseries]
      \setenumerate[2]{leftmargin=*,labelsep=0em}
      \item}
    {\end{enumerate}}
  \renewcommand{\reponseparams}{{\thechapter.\theexercice}}

  %% Listes de commandes
  \newenvironment{ttscript}[1]{%
    \begin{list}{}{%
        \setlength{\labelsep}{1.5ex}
        \settowidth{\labelwidth}{\fbox{\code{#1}}}
        \setlength{\leftmargin}{\labelwidth}
        \addtolength{\leftmargin}{\labelsep}
        \setlength{\parsep}{0.5ex plus0.2ex minus0.2ex}
        \setlength{\itemsep}{0.3ex}
        \renewcommand{\makelabel}[1]{\fbox{##1}\hfill}}}
    {\end{list}}

  %% Chapitre Opérateurs: environnements pour placer côte à côte des
  %% définitions de fonctions (utilisant ttscript ci-dessus) et un
  %% petit exemple.
  \newenvironment{operateur}[1]{%
    \noindent
    \begin{minipage}[t]{0.65\linewidth}
      \begin{ttscript}{#1}}
    {\end{ttscript}\end{minipage}\hfill}
  \newenvironment{miniexemple}{%
    \begin{minipage}[t]{0.3\linewidth}}
    {\end{minipage}}

  %%% Listes de structures de contrôle
  \newenvironment{struclist}{%
    \begin{description}[\breaklabel\setlabelstyle{\mdseries\ttfamily}%
      \setleftmargin{\parindent}]}
    {\end{description}}

  %%% Remarques importantes
  \newenvironment{important}{%
    \begin{framed}%
      \noindent
      \begin{minipage}{0.1\linewidth}
        \raisebox{-0.5em}[0em][0em]{\LARGE\danger}
      \end{minipage}
      \begin{minipage}[t]{0.85\linewidth}}
      {\end{minipage}%
    \end{framed}}

  %%% =============================================
  %%%  Paramètres pour les sections de code source
  %%% =============================================
  \lstloadlanguages{R}
  \lstdefinelanguage{Renhanced}[]{R}{%
    morekeywords={colMeans,colSums,head,is.na,is.null,mapply,ms,na.rm,%
      nlmin,replicate,row.names,rowMeans,rowSums,sys.time,system.time,%
      tail,which.max,which.min},
    deletekeywords={c},
    alsoletter={.\%},%
    alsoother={:_\$}}
  \lstset{language=Renhanced,
    extendedchars=true,
    inputencoding=utf8/latin1,
    basicstyle=\small\ttfamily,
    commentstyle=\color{comments}\slshape,
    keywordstyle=\mdseries,
    showstringspaces=false,
    index=[1][keywords],
    indexstyle=\indexfonction}

  %%% =======
  %%%  Index
  %%% =======
  \newcommand{\bfhyperpage}[1]{\textbf{\hyperpage{#1}}}
  \renewcommand{\preindexhook}{%
    Les numéros de page en caractères gras indiquent les pages où les
    concepts sont introduits, définis ou expliqués.\vskip\onelineskip}
  \newcommand{\Index}[1]{\index{#1|bfhyperpage}}
  \newcommand{\indexargument}[1]{\index{#1@\code{#1}}}
  \newcommand{\Indexargument}[1]{\Index{#1@\code{#1}}}
  \newcommand{\indexattribut}[1]{\index{#1@\code{#1} (attribut)}}
  \newcommand{\Indexattribut}[1]{\Index{#1@\code{#1} (attribut)}}
  \newcommand{\indexclasse}[1]{\index{#1@\code{#1} (classe)}}
  \newcommand{\Indexclasse}[1]{\Index{#1@\code{#1} (classe)}}
  \newcommand{\indexfonction}[1]{\index{#1@\code{#1}}}
  \newcommand{\Indexfonction}[1]{\Index{#1@\code{#1}}}
  \newcommand{\indexmode}[1]{\index{#1@\code{#1} (mode)}}
  \newcommand{\Indexmode}[1]{\Index{#1@\code{#1} (mode)}}
  \newcommand{\indexobjet}[1]{\index{#1@\code{#1}}}
  \newcommand{\Indexobjet}[1]{\Index{#1@\code{#1}}}
  \newcommand{\indexemacs}[1]{\index{Emacs!#1@\texttt{#1}}}
  \newcommand{\indexess}[1]{\index{ESS!#1@\texttt{#1}}}

  \newcommand{\attribut}[1]{\code{#1}\indexattribut{#1}}
  \newcommand{\Attribut}[1]{\code{#1}\Indexattribut{#1}}
  \newcommand{\argument}[1]{\code{#1}\indexargument{#1}}
  \newcommand{\Argument}[1]{\code{#1}\Indexargument{#1}}
  \newcommand{\classe}[1]{\code{#1}\indexclasse{#1}}
  \newcommand{\Classe}[1]{\code{#1}\Indexclasse{#1}}
  \newcommand{\fonction}[1]{\code{#1}\indexfonction{#1}}
  \newcommand{\Fonction}[1]{\code{#1}\Indexfonction{#1}}
  \newcommand{\mode}[1]{\code{#1}\indexmode{#1}}
  \newcommand{\Mode}[1]{\code{#1}\Indexmode{#1}}
  \newcommand{\objet}[1]{\code{#1}\indexobjet{#1}}
  \newcommand{\Objet}[1]{\code{#1}\Indexobjet{#1}}
  \newcommand{\emacs}[1]{\code{#1}\indexemacs{#1}}
  \newcommand{\ess}[1]{\code{#1}\indexess{#1}}
  \makeindex

  %%% Sous-figures
  \newsubfloat{figure}

  %%% Style de la bibliographie
  \bibliographystyle{francais}

  %%% Aide pour la césure
  \hyphenation{con-sole}

  \includeonly{pagegarde,notices}

\begin{document}

\frontmatter

\pagestyle{empty}

%% Page couverture avant. Il faut modifier la largeur des graphiques
%% puisque Sweave la règle à 0.8\textwidth.
\setkeys{Gin}{width=\paperwidth}
\includepdf[pages=1]{couvertures-partie_1}
\setkeys{Gin}{width=0.8\textwidth}
\cleardoublepage

\begin{adjustwidth*}{-12mm}{-72mm}
  \sffamily
  \raggedright
  \vspace*{-17mm}
  \thetitle \\
  \vspace*{20mm}
  \theparttitle \\
  \vspace*{32mm}
  \theauthor \\
  \vspace*{\fill}
  \thedate
\end{adjustwidth*}

%%% Local Variables:
%%% mode: latex
%%% TeX-master: "methodes_numeriques-partie_3"
%%% coding: utf-8
%%% End:

\clearpage

\begingroup
\calccentering{\unitlength}
\begin{adjustwidth*}{\unitlength}{-\unitlength}
  \setlength{\parindent}{0pt}
  \setlength{\parskip}{\baselineskip}

  {\textcopyright} {\year} Vincent Goulet \\

  \includegraphics[height=7mm,keepaspectratio=true]{by-sa}\\%
Cette création est mise à disposition selon le contrat
\href{http://creativecommons.org/licenses/by-sa/2.5/ca/}{%
  Paternité-Partage à l'identique 2.5 Canada} de Creative Commons. En
vertu de ce contrat, vous êtes libre de:
\begin{itemize}
\item \textbf{partager} —-- reproduire, distribuer et communiquer
  l'{\oe}uvre;
\item \textbf{remixer} —-- adapter l'{\oe}uvre;
\item utiliser cette {\oe}uvre à des fins commerciales.
\end{itemize}
Selon les conditions suivantes:

\begin{tabularx}{\linewidth}{@{}lX@{}}
  \raisebox{-9mm}[0mm][13mm]{%
    \includegraphics[height=11mm,keepaspectratio=true]{by}} &
  \textbf{Attribution} —-- Vous devez attribuer l'{\oe}uvre de la
  manière indiquée par l'auteur de l'{\oe}uvre ou le titulaire des
  droits (mais pas d'une manière qui suggérerait qu'ils vous
  soutiennent ou
  approuvent votre utilisation de l'{\oe}uvre). \\
  \raisebox{-9mm}{\includegraphics[height=11mm,keepaspectratio=true]{sa}}
  & \textbf{Partage à l'identique} --— Si vous modifiez, transformez
  ou adaptez cette {\oe}uvre, vous n'avez le droit de distribuer votre
  création que sous une licence identique ou similaire à celle-ci.
\end{tabularx}


  \textbf{Code source} \\
  \begin{tabularx}{1.0\linewidth}{@{}Xl@{}}
    Code informatique des sections d'exemples & \href{http://libre.act.ulaval.ca/fileadmin/Portail_libre/ACT-2002/Notes\%20de\%20cours/code-partie_1.zip}{\downloadbutton} \\
    \addlinespace[3pt]
    Sorties du code informatique & \href{http://libre.act.ulaval.ca/fileadmin/Portail_libre/ACT-2002/Notes\%20de\%20cours/code-partie_1-sorties.zip}{\downloadbutton} \\
    \addlinespace[3pt]
    Code source du document & \href{https://svn.fsg.ulaval.ca/svn-pub/vgoulet/documents/methodes_numeriques/}{\browsebutton}
  \end{tabularx}

  \textbf{Couverture} \\
  Le reptile en couverture est un caméléon panthère (\emph{Furcifer
    pardalis}) originaire de Madagascar. Il s'agit d'un des plus
  grands caméléons existants, la taille du mâle pouvant atteindre
  55~cm, queue comprise.

  Crédit photo: Maria Stenzel, National Geographic Society
\end{adjustwidth*}
\endgroup

%%% Local Variables:
%%% mode: latex
%%% TeX-master: "methodes_numeriques-partie_1"
%%% coding: utf-8
%%% End:

\clearpage

\pagestyle{companion}

\chapter*{Introduction}
\addcontentsline{toc}{chapter}{Introduction}
\markboth{Introduction}{Introduction}

Il existe de multiples ouvrages traitant de l'environnement
statistique R. Dans la majorité des cas, toutefois, le logiciel est
présenté dans le cadre d'applications statistiques spécifiques. Ce
document se concentre plutôt sur l'apprentissage du langage de
programmation sous-jacent aux diverses fonctions statistiques, langage
lui aussi nommé R.

Chaque chapitre présente en rafale plusieurs éléments de théorie avec
généralement peu d'exemples. La lecture d'un chapitre permet donc
d'acquérir rapidement plusieurs nouvelles connaissances sur le langage
R. Cependant, pour compléter son apprentissage, le lecteur devra aussi
étudier attentivement et, surtout, exécuter ligne par ligne le code R
fourni dans les sections d'exemples à la fin des chapitres (sauf un).
Ces sections d'exemples couvrent l'essentiel des concepts présentés
dans les chapitres et les complémentent souvent. L'étude de ces
sections fait partie intégrante de l'apprentissage du langage R.

Le code des sections d'exemples est disponible dans le site du cours.
Nous fournissons également des fichiers de sortie contenant les
résultats de chacune des expressions.

Certains exemples et exercices font référence à des concepts de base
de la théorie des probabilités et des mathématiques financières. Les
contextes actuariels demeurent néanmoins peu nombreux et ne devraient
généralement pas dérouter le lecteur pour qui ces notions sont moins
familières. Les réponses de tous les exercices se trouvent en annexe.

On trouvera également en annexe une brève introduction à l'éditeur de
texte GNU~Emacs et au mode ESS, ainsi qu'une présentation sur
l'administration d'une bibliothèque de packages R.

Je tiens à remercier M.~Mathieu Boudreault pour sa collaboration
dans la rédaction des exercices.

%%% Local Variables:
%%% mode: latex
%%% TeX-master: "methodes_numeriques-partie_1"
%%% coding: utf-8
%%% End:

\cleartorecto
\tableofcontents*

\mainmatter
\include{presentation}
\include{bases}
\include{operateurs}
\include{exemples}
\include{fonctions}
\include{avance}

\appendix
\include{emacs+ess}
\include{packages}
\chapter*{Réponses des exercices}
\label{reponses}
\addcontentsline{toc}{chapter}{Réponses des exercices}
\markboth{Réponses des exercices}{Réponses des exercices}

\input{reponses-bases}
\input{reponses-operateurs}
\input{reponses-exemples}
\input{reponses-fonctions}
\input{reponses-avance}

%%% Local Variables:
%%% mode: latex
%%% TeX-master: "methodes_numeriques-partie_1"
%%% coding: utf-8
%%% End:
     % différent de Introduction à la...

\bibliography{r,stat,informatique}

\cleardoublepage
\printindex

%%% Copyright (C) 2018 Vincent Goulet
%%%
%%% Ce fichier fait partie du projet
%%% «Méthodes numériques en actuariat avec R»
%%% http://github.com/vigou3/methodes-numeriques-en-actuariat
%%%
%%% Cette création est mise à disposition selon le contrat
%%% Attribution-Partage dans les mêmes conditions 4.0
%%% International de Creative Commons.
%%% http://creativecommons.org/licenses/by-sa/4.0/

\vspace*{\fill}

\begingroup
\calccentering{\unitlength}
\begin{adjustwidth*}{\unitlength}{-\unitlength}
  \begin{flushleft}
    \small %
    Ce document a été produit avec le système de mise en page
    {\XeLaTeX}. Le texte principal est en Lucida Bright~OT 11~points,
    les mathématiques en Lucida Bright Math~OT, le code informatique
    en Lucida Grande Mono~DK et les titres en Adobe Myriad~Pro. Des
    icônes proviennent de la police Font~Awesome. Les graphiques ont
    été réalisés avec R.
  \end{flushleft}
\end{adjustwidth*}
\endgroup
\vfill


\cleardoublepage
\cleartoverso

%% Page couverture arrière.
\setkeys{Gin}{width=\paperwidth}
\includepdf[pages=2]{couvertures-partie_1}

\end{document}

%%% Local Variables:
%%% mode: latex
%%% TeX-master: t
%%% coding: utf-8
%%% End:
