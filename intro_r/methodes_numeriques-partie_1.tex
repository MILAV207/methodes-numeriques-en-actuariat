\documentclass[letterpaper,11pt,x11names,english,french]{memoir}
  \usepackage{natbib,url}
  \usepackage{babel}
  \usepackage[autolanguage]{numprint}
  \usepackage{actuarialangle,amsmath,amsthm}
  \usepackage[noae]{Sweave}
  \usepackage{framed}                  % env. snugshade*
  \usepackage{paralist}
  \usepackage[shortlabels]{enumitem}
  \usepackage{textpos}
  \usepackage{graphicx}
  \usepackage{manfnt}                  % \mantriangleright (puce)
  \usepackage{metalogo}                % \XeLaTeX logo
  \usepackage{applekeys}               % touches Mac
  \usepackage{fontawesome}             % icônes \fa*
  \usepackage{answers}                 % exercices et solutions
  \usepackage{listings}                % code informatique
  \usepackage{pdfpages}                % couvertures
  \usepackage{xr}

  %%% ===================
  %%%  Style du document
  %%% ===================

  %% Polices de caractères
  \usepackage{fontspec}
  \usepackage{unicode-math}
  \defaultfontfeatures{Ligatures=TeX,Scale=0.92}
  \setmainfont[Numbers=OldStyle]{Lucida Bright OT}
  \setmathfont{Lucida Bright Math OT}
  \setmonofont{Lucida Grande Mono DK}
  \setsansfont[Scale=1.0,Numbers=OldStyle]{Myriad Pro}
  \newfontfamily\fullcaps[Letters=Uppercase,Numbers=Uppercase]{Myriad Pro}
  \usepackage[babel=true]{microtype}
  \usepackage{icomma}

  %% Couleurs
  \usepackage{xcolor}
  \definecolor{comments}{rgb}{0.7,0,0}          % rouge foncé
  \definecolor{link}{rgb}{0,0.4,0.6}            % ~RoyalBlue de dvips
  \definecolor{url}{rgb}{0.6,0,0}               % rouge-brun
  \definecolor{citation}{rgb}{0,0.5,0}          % vert foncé
  \definecolor{shadecolor}{named}{LightYellow1} % code source

  %% Hyperliens
  \usepackage{hyperref}
  \hypersetup{colorlinks, linktocpage,
    urlcolor=url, linkcolor=link, citecolor=citation,
    bookmarksopen, bookmarksnumbered, bookmarksdepth=subsubsection,
    pdfauthor={Vincent Goulet}}
  \setlength{\XeTeXLinkMargin}{1pt}

  %% Étiquettes de \autoref (redéfinitions compatibles avec babel).
  %% Attention! Les % à la fin des lignes sont importants sinon des
  %% blancs apparaissent dès que la commande \selectlanguage est
  %% utilisée... comme dans la bibliographie, par exemple.
  \addto\extrasfrench{%
    \def\algorithmeautorefname{algorithme}%
    \def\appendixautorefname{annexe}%
    \def\definitionautorefname{définition}%
    \def\figureautorefname{figure}%
    \def\exempleautorefname{exemple}%
    \def\exerciceautorefname{exercice}%
    \def\subfigureautorefname{figure}%
    \def\subsectionautorefname{section}%
    \def\subtableautorefname{tableau}%
    \def\tableautorefname{tableau}%
    \def\thmautorefname{théorème}%
  }

  %% Table des matières (inspirée de classicthesis.sty)
  \renewcommand{\cftchapterleader}{\hspace{1.5em}}
  \renewcommand{\cftchapterafterpnum}{\cftparfillskip}
  \renewcommand{\cftsectionleader}{\hspace{1.5em}}
  \renewcommand{\cftsectionafterpnum}{\cftparfillskip}

  %% Titres des chapitres
  \chapterstyle{hangnum}
  \renewcommand{\chaptitlefont}{\normalfont\Huge\sffamily\bfseries\raggedright}

  %% Marges, entêtes et pieds de page
  \setlength{\marginparsep}{7mm}
  \setlength{\marginparwidth}{13mm}
  \setlength{\headwidth}{\textwidth}
  \addtolength{\headwidth}{\marginparsep}
  \addtolength{\headwidth}{\marginparwidth}

  %% Titres des sections et sous-sections
  \setsecheadstyle{\normalfont\Large\sffamily\bfseries\raggedright}
  \setsubsecheadstyle{\normalfont\large\sffamily\bfseries\raggedright}
  \maxsecnumdepth{subsection}
  \setsecnumdepth{subsection}

  %% Listes. Paramétrage avec enumitem.
  \setlist[enumerate]{leftmargin=*,align=left}
  \setlist[enumerate,2]{label=\alph*),labelsep=*,leftmargin=1.5em}
  \setlist[enumerate,3]{label=\roman*),labelsep=*,leftmargin=1.5em,align=right}
  \setlist[itemize]{leftmargin=*,align=left}

  %% Options de babel
  \frenchbsetup{CompactItemize=false,%
    ThinSpaceInFrenchNumbers=true,
    ItemLabeli=\mantriangleright,
    ItemLabelii=\textendash}
  \addto\captionsfrench{\def\figurename{{\scshape Fig.}}}
  \addto\captionsfrench{\def\tablename{{\scshape Tab.}}}

  %% Sections de code source
  \lstloadlanguages{R}
  \lstset{language=R,
    extendedchars=true,
    basicstyle=\small\ttfamily\NoAutoSpacing,
    commentstyle=\color{comments}\slshape,
    keywordstyle=\mdseries,
    showstringspaces=false}

  %%% =========================
  %%%  Nouveaux environnements
  %%% =========================

  %% Environnements d'exemples et al.
  \theoremstyle{plain}
  \newtheorem{algorithme}{Algorithme}[chapter]
  \newtheorem{thm}{Théorème}[chapter]

  \theoremstyle{definition}
  \newtheorem{exemple}{Exemple}[chapter]
  \newtheorem{definition}{Définition}[chapter]
  \newtheorem*{astuce}{Astuce}

  \theoremstyle{remark}
  \newtheorem*{remarque}{Remarque}
  \newtheorem*{remarques}{Remarques}
  \newenvironment{rem}{\begin{remarque} \mbox{}}{\end{remarque}}
  \newenvironment{rems}{\begin{remarques} \mbox{}}{\end{remarques}}

  %% Listes d'objectifs au début des chapitres
  \newenvironment{objectifs}{%
    \noindent
    \definecolor{shadecolor}{named}{black}
    \begin{framed}
      \vspace{-1.33\baselineskip}
      \begin{shaded}
        \noindent\sffamily\bfseries\textcolor{white}{Objectifs du chapitre}
      \end{shaded}
      \vspace{-0.6\baselineskip}
      \begin{itemize}[nosep]
        \small\sffamily}
      {\end{itemize}
    \end{framed}}

  %% Environnements de Sweave. Les environnements Sinput et Soutput
  %% utilisent Verbatim (de fancyvrb). On les réinitialise pour
  %% enlever la configuration par défaut de Sweave, puis on réduit
  %% l'écart entre les blocs Sinput et Soutput. L'environnement Schunk
  %% utilise snugshade* (de framed).
  \DefineVerbatimEnvironment{Sinput}{Verbatim}{}
  \DefineVerbatimEnvironment{Soutput}{Verbatim}{}
  \fvset{listparameters={\setlength{\topsep}{0pt}}}
  \renewenvironment{Schunk}{%
    \setlength{\topsep}{0pt}\begin{snugshade*}}%
    {\end{snugshade*}}

  %% Exercices et réponses
  \Newassociation{sol}{solution}{solutions}
  \Newassociation{rep}{reponse}{reponses}
  \newcounter{exercice}[chapter]
  \renewcommand{\theexercice}{\thechapter.\arabic{exercice}}
  \newenvironment{exercice}[1][]{%
    \begin{list}{}{%
        \refstepcounter{exercice}
        \ifthenelse{\equal{#1}{nosol}}{%
          \renewcommand{\makelabel}{\bfseries\theexercice}}{%
          \hypertarget{ex:\theexercice}{}
          \Writetofile{solutions}{\protect\hypertarget{sol:\theexercice}{}}
          \renewcommand{\makelabel}{%
            \bfseries\protect\hyperlink{sol:\theexercice}{\theexercice}}}
        \settowidth{\labelwidth}{\bfseries\theexercice}
        \setlength{\leftmargin}{\labelwidth}
        \addtolength{\leftmargin}{\labelsep}
        \setlist[enumerate,1]{label=\alph*),labelsep=*,leftmargin=1.5em}
        \setlist[enumerate,2]{label=\roman*),labelsep=0.5em,align=right}}
      \item}
    {\end{list}}
  \renewenvironment{solution}[1]{%
    \begin{list}{}{%
        \renewcommand{\makelabel}{%
          \bfseries\protect\hyperlink{ex:#1}{#1}}
        \settowidth{\labelwidth}{\bfseries #1}
        \setlength{\leftmargin}{\labelwidth}
        \addtolength{\leftmargin}{\labelsep}
        \setlist[enumerate,1]{label=\alph*),labelsep=*,leftmargin=1.5em}
        \setlist[enumerate,2]{label=\roman*),labelsep=0.5em,align=right}}
      \item}
    {\end{list}}
  \renewenvironment{reponse}[1]{%
    \begin{enumerate}[label=\textbf{#1}]
      \item}
    {\end{enumerate}}

  %% Remarques importantes
  \newenvironment{important}{%
    \begin{framed}%
      \noindent
      \begin{minipage}{0.1\linewidth}
        \raisebox{-1.5em}[0em][0em]{\HUGE\faExclamationCircle}
      \end{minipage}
      \begin{minipage}[t]{0.88\linewidth}}
      {\end{minipage}%
    \end{framed}}

  %% Informations
  \newenvironment{information}{%
    \begin{framed}%
      \noindent
      \begin{minipage}{0.1\linewidth}
        \raisebox{-1.5em}[0em][0em]{\HUGE\faInfoCircle}
      \end{minipage}
      \begin{minipage}[t]{0.88\linewidth}}
      {\end{minipage}%
    \end{framed}}

  %% Remarques spécifiques OS X
  \newenvironment{osx}{%
    \begin{framed}%
      \noindent
      \begin{minipage}{0.1\linewidth}
        \raisebox{-1.5em}[0em][0em]{\HUGE\faApple}
      \end{minipage}
      \begin{minipage}[t]{0.88\linewidth}}
      {\end{minipage}%
    \end{framed}}

  %% Changements au fil de la lecture
  \newenvironment{gotoR}{%
    \begin{framed}%
      \noindent
      \begin{minipage}{0.07\linewidth}
        \raisebox{-0.7em}[0em][0em]{\Large\faFastForward}
      \end{minipage}
      \begin{minipage}[t]{0.88\linewidth}\sffamily}
      {\end{minipage}%
    \end{framed}}

  %% Redéfinition de l'environnement de matrices de amsmath pour
  %% aligner les colonnes à droite. Pris dans
  %% <http://texblog.net/latex-archive/maths/matrix-align-left-right/>
  \makeatletter
  \renewcommand*\env@matrix[1][r]{\hskip -\arraycolsep
    \let\@ifnextchar\new@ifnextchar
    \array{*\c@MaxMatrixCols #1}}
  \makeatother

  %%% =====================
  %%%  Nouvelles commandes
  %%% =====================

  %% Noms de fonctions, code, etc.
  \newcommand{\code}[1]{\texttt{#1}}
  \newcommand{\pkg}[1]{\textbf{#1}}

  %% Indications de capsule vidéo
  \newcommand{\capsule}[2]{\href{#1}{#2}\marginpar{%
      \href{#1}{\raisebox{-0.5em}[0em][0em]{\HUGE\faYoutubePlay}}}}

  %% «Boutons» de la page de notices
  \newcommand{\downloadbutton}{%
    \framebox[40mm][l]{%
      \rule[-5pt]{0mm}{16pt}%
      \makebox[7mm]{\raisebox{-2.5pt}{\LARGE\faDownload}}\;%
      {\sffamily Télécharger}}}
  \newcommand{\browsebutton}{%
    \framebox[40mm][l]{%
      \rule[-5pt]{0mm}{16pt}%
      \makebox[7mm]{\raisebox{-3pt}{\LARGE\faExternalLink}}\;%
      {\sffamily Accéder au dépôt}}}

  %% Raccourcis usuels vg
  \newcommand{\pt}{{\scriptscriptstyle \Sigma}}
  \newcommand{\abs}[1]{\lvert #1 \rvert}
  \newcommand{\norme}[1]{\lVert #1 \rVert}
  \newcommand{\mat}[1]{\mathbf{#1}}
  \newcommand{\diag}{\operatorname{diag}}
  \newcommand{\Esp}[1]{E\! \left[ #1 \right]}
  \newcommand{\esp}[1]{E [ #1 ]}
  \newcommand{\Var}[1]{\operatorname{Var}\! \left[ #1 \right]}
  \newcommand{\var}[1]{\operatorname{Var} [ #1 ]}
  \newcommand{\Prob}[1]{\operatorname{Pr}\! \left[ #1 \right]}
  \newcommand{\prob}[1]{\operatorname{Pr} [ #1 ]}
  \newcommand{\R}{\mathbb{R}}   % ensemble des réels

  %% Traitement du titre de partie
  \makeatletter
  \newcommand{\@parttitle}{}
  \newcommand{\parttitle}[1]{\renewcommand{\@parttitle}{#1}}
  \newcommand{\theparttitle}{\@parttitle}
  \makeatother

  %%% =======
  %%%  Varia
  %%% =======

  %% Sous-tableaux et figures
  \newsubfloat{table}
  \newsubfloat{figure}

  %% Style de la bibliographie
  \bibliographystyle{francais}

  %% Aide pour la césure
  \hyphenation{%
    con-gru-en-tiels
    con-naî-tre
    con-sole
    cons-tante
    con-tenu
    con-trôle
    hexa-dé-ci-mal
    nom-bre
    puis-que
  }

  %%% ======================================
  %%%  Page titre (sauf titre de la partie)
  %%% ======================================
  \renewcommand{\year}{2016}
  \title{%
    \bfseries\fontsize{42}{33}\selectfont {\fullcaps ACT 2002} \\
    \mdseries\fontsize{32}{33}\selectfont Méthodes numériques \\
                                          en actuariat}
  \author{%
    \bfseries\fontsize{16}{20}\selectfont Vincent Goulet \\
    \mdseries\fontsize{14}{18}\selectfont Professeur titulaire \textbar\
                                          École d'actuariat \textbar\
                                          Université Laval}
  \date{%
    \mdseries\fontsize{14}{18}\selectfont Notes de cours \textbar\
                                          Exercices \textemdash\
                                          édition \year}

  \usepackage{wasysym}         % \leadsto

  %%%  Titre du document
  \parttitle{%
    \bfseries\fontsize{36}{36}\selectfont Partie I \\
    \mdseries\fontsize{32}{36}\selectfont Programmation en R}
  \hypersetup{pdftitle={Méthodes numériques en actuariat - Partie I
      Programmation en R}}

  %%% Listes de commandes
  \newenvironment{ttscript}[1]{%
    \begin{list}{}{%
        \setlength{\labelsep}{1.5ex}
        \settowidth{\labelwidth}{\fbox{\code{#1}}}
        \setlength{\leftmargin}{\labelwidth}
        \addtolength{\leftmargin}{\labelsep}
        \setlength{\parsep}{0.5ex plus0.2ex minus0.2ex}
        \setlength{\itemsep}{0.3ex}
        \renewcommand{\makelabel}[1]{\fbox{\vphantom{|}##1}\hfill}}}
    {\end{list}}

  %%% Chapitre Opérateurs: l'espacement entre les expressions R et
  %%% leur sortie est réduit pour les exemples de fonctions.
  \newlength{\compactsep} \setlength{\compactsep}{-0.5ex}
  \newlength{\normalsep}  \setlength{\normalsep}{\topsep}

  %%% Listes de structures de contrôle
  \newenvironment{struclist}{%
    \begin{description}[style=nextline,font=\mdseries\ttfamily]}
    {\end{description}}

  %%% Paramètres additionels pour sections de code source
  \lstdefinelanguage{Renhanced}[]{R}{%
    morekeywords={colMeans,colSums,head,is.na,is.null,mapply,ms,na.rm,%
      nlmin,replicate,row.names,rowMeans,rowSums,sys.time,system.time,%
      tail,which.max,which.min},
    deletekeywords={c,start},
    alsoletter={.\%},%
    alsoother={:_\$}}
  \lstset{language=Renhanced,
    index=[1][keywords],
    indexstyle=\indexfonction}

  %%% Index
  \newcommand{\bfhyperpage}[1]{\textbf{\hyperpage{#1}}}
  \renewcommand{\preindexhook}{%

    Les numéros de page en caractères gras indiquent les pages où les
    concepts sont introduits, définis ou expliqués.\vskip\onelineskip}
  \newcommand{\Index}[1]{\index{#1|bfhyperpage}}
  \newcommand{\indexargument}[1]{\index{#1@\code{#1}}}
  \newcommand{\Indexargument}[1]{\Index{#1@\code{#1}}}
  \newcommand{\indexattribut}[1]{\index{#1@\code{#1} (attribut)}}
  \newcommand{\Indexattribut}[1]{\Index{#1@\code{#1} (attribut)}}
  \newcommand{\indexclasse}[1]{\index{#1@\code{#1} (classe)}}
  \newcommand{\Indexclasse}[1]{\Index{#1@\code{#1} (classe)}}
  \newcommand{\indexfonction}[1]{\index{#1@\code{#1}}}
  \newcommand{\Indexfonction}[1]{\Index{#1@\code{#1}}}
  \newcommand{\indexmode}[1]{\index{#1@\code{#1} (mode)}}
  \newcommand{\Indexmode}[1]{\Index{#1@\code{#1} (mode)}}
  \newcommand{\indexobjet}[1]{\index{#1@\code{#1}}}
  \newcommand{\Indexobjet}[1]{\Index{#1@\code{#1}}}

  \newcommand{\attribut}[1]{\code{#1}\indexattribut{#1}}
  \newcommand{\Attribut}[1]{\code{#1}\Indexattribut{#1}}
  \newcommand{\argument}[1]{\code{#1}\indexargument{#1}}
  \newcommand{\Argument}[1]{\code{#1}\Indexargument{#1}}
  \newcommand{\classe}[1]{\code{#1}\indexclasse{#1}}
  \newcommand{\Classe}[1]{\code{#1}\Indexclasse{#1}}
  \newcommand{\fonction}[1]{\code{#1}\indexfonction{#1}}
  \newcommand{\Fonction}[1]{\code{#1}\Indexfonction{#1}}
  \newcommand{\mode}[1]{\code{#1}\indexmode{#1}}
  \newcommand{\Mode}[1]{\code{#1}\Indexmode{#1}}
  \newcommand{\objet}[1]{\code{#1}\indexobjet{#1}}
  \newcommand{\Objet}[1]{\code{#1}\Indexobjet{#1}}
  \makeindex

  \includeonly{emacs+ess,rstudio}

\begin{document}

\frontmatter

\pagestyle{empty}

%% Page couverture avant. Il faut modifier la largeur des graphiques
%% puisque Sweave la règle à 0.8\textwidth.
\setkeys{Gin}{width=\paperwidth}
\includepdf[pages=1]{couvertures-partie_1}
\setkeys{Gin}{width=0.8\textwidth}
\cleardoublepage

\include{frontispice}
\clearpage

\begingroup
\calccentering{\unitlength}
\begin{adjustwidth*}{\unitlength}{-\unitlength}
  \setlength{\parindent}{0pt}
  \setlength{\parskip}{\baselineskip}

  {\textcopyright} {\year} Vincent Goulet \\

  \includegraphics[height=7mm,keepaspectratio=true]{by-sa}\\%
Cette création est mise à disposition selon le contrat
\href{http://creativecommons.org/licenses/by-sa/2.5/ca/}{%
  Paternité-Partage à l'identique 2.5 Canada} de Creative Commons. En
vertu de ce contrat, vous êtes libre de:
\begin{itemize}
\item \textbf{partager} —-- reproduire, distribuer et communiquer
  l'{\oe}uvre;
\item \textbf{remixer} —-- adapter l'{\oe}uvre;
\item utiliser cette {\oe}uvre à des fins commerciales.
\end{itemize}
Selon les conditions suivantes:

\begin{tabularx}{\linewidth}{@{}lX@{}}
  \raisebox{-9mm}[0mm][13mm]{%
    \includegraphics[height=11mm,keepaspectratio=true]{by}} &
  \textbf{Attribution} —-- Vous devez attribuer l'{\oe}uvre de la
  manière indiquée par l'auteur de l'{\oe}uvre ou le titulaire des
  droits (mais pas d'une manière qui suggérerait qu'ils vous
  soutiennent ou
  approuvent votre utilisation de l'{\oe}uvre). \\
  \raisebox{-9mm}{\includegraphics[height=11mm,keepaspectratio=true]{sa}}
  & \textbf{Partage à l'identique} --— Si vous modifiez, transformez
  ou adaptez cette {\oe}uvre, vous n'avez le droit de distribuer votre
  création que sous une licence identique ou similaire à celle-ci.
\end{tabularx}


  \textbf{Code source} \\
  \begin{tabularx}{1.0\linewidth}{@{}Xl@{}}
    Code informatique des sections d'exemples & \href{http://libre.act.ulaval.ca/fileadmin/Portail_libre/ACT-2002/Notes\%20de\%20cours/code-partie_1.zip}{\downloadbutton} \\
    \addlinespace[3pt]
    Sorties du code informatique & \href{http://libre.act.ulaval.ca/fileadmin/Portail_libre/ACT-2002/Notes\%20de\%20cours/code-partie_1-sorties.zip}{\downloadbutton} \\
    \addlinespace[3pt]
    Code source du document & \href{https://svn.fsg.ulaval.ca/svn-pub/vgoulet/documents/methodes_numeriques/}{\browsebutton}
  \end{tabularx}

  \textbf{Couverture} \\
  Le reptile en couverture est un caméléon panthère (\emph{Furcifer
    pardalis}) originaire de Madagascar. Il s'agit d'un des plus
  grands caméléons existants, la taille du mâle pouvant atteindre
  55~cm, queue comprise.

  Crédit photo: Maria Stenzel, National Geographic Society
\end{adjustwidth*}
\endgroup

%%% Local Variables:
%%% mode: latex
%%% TeX-master: "methodes_numeriques-partie_1"
%%% coding: utf-8
%%% End:
              % NB: différent de Introduction à la...
\clearpage

\pagestyle{companion}

\chapter*{Introduction}
\addcontentsline{toc}{chapter}{Introduction}
\markboth{Introduction}{Introduction}


L'étude de ce document implique quelques allers-retours entre le texte
et les sections de code informatique présentes dans chaque chapitre.
Les changements dans le fil de la lecture sont clairement indiqués
dans le texte par des mentions mises en évidence par le symbole
{\color{darkred}\noway}.

Chaque chapitre comporte des d'exercices. Les réponses de ceux-ci se
trouvent à la fin de chacun des chapitres et les solutions complètes
en annexe.

Je tiens à souligner la précieuse collaboration de MM.~Mathieu
Boudreault, Sébastien Auclair et Louis-Philippe Pouliot lors de la
rédaction des exercices et des solutions. Je remercie également
Mmes~Marie-Pier Laliberté et Véronique Tardif pour l'infographie des
pages couvertures.

%%% Local Variables:
%%% mode: latex
%%% TeX-master: "methodes_numeriques-partie_4"
%%% End:
         % NB: différent de Introduction à la...

\cleartorecto
\tableofcontents*

%% Vignette tirée de xkcd.com
\cleartoverso
\thispagestyle{empty}
\begin{vplace}[0.45]
  \centering
  \setkeys{Gin}{width=\textwidth}
  \begin{minipage}{351pt}
    \includegraphics{compiling.png} \\
    \footnotesize\sffamily%
    Tiré de \href{http://xkcd.com/303/}{XKCD.com}
  \end{minipage}
  \setkeys{Gin}{width=0.8\textwidth}
\end{vplace}

\mainmatter

\include{presentation}
\include{bases}
\include{operateurs}
\include{exemples}
\include{fonctions}
\include{avance}

\appendix
\include{emacs+ess}
\chapter{RStudio: une introduction}
\label{rstudio}


\section{Installation}
\label{rstudio:installation}



\section{Description sommaire}
\label{rstudio:description}


\begin{figure}[t]
  %% Capture d'écran
  \includegraphics{emacswindow-screenshot}

  %% Identification du minibuffer
  \begin{textblock}{0.35}(7.35,-0.08)
    \LARGE$\leadsto$
  \end{textblock}
  \begin{textblock}{1.75}(7.85,-0.1)
    \small \emph{Minibuffer}
  \end{textblock}

  %% Identification de la barre de menu
  \begin{textblock}{0.35}(7.35,-6.58)
    \LARGE$\leadsto$
  \end{textblock}
  \begin{textblock}{1.75}(7.85,-6.6)
    \small Barre de menu
  \end{textblock}

  %% Identification du buffer
  \begin{textblock}{0.35}(7.35,-3.38)
    \LARGE$\leadsto$
  \end{textblock}
  \begin{textblock}{1.75}(7.85,-3.4)
    \small \emph{Buffer}
  \end{textblock}

  %% Identification de la mode line
  \begin{textblock}{0.35}(7.35,-0.28)
    \LARGE$\leadsto$
  \end{textblock}
  \begin{textblock}{1.75}(7.85,-0.3)
    \small Ligne de mode
  \end{textblock}
  \caption{Fenêtre GNU~Emacs et ses différentes parties au lancement
    de l'application sous Mac OS~X. Sous Windows et Linux, la barre de
    menu se trouve à l'intérieur de la fenêtre.}
  \label{fig:ess:emacswindow}
\end{figure}



\begin{figure}[t]
  \begin{osx}
    Par défaut sous Mac OS~X, la touche \code{Meta} est assignée à
    \code{Option} (\optkey). Sur les claviers français, cela empêche
    d'accéder à certains caractères spéciaux tels que [, ], \{ ou \}.

    Une solution consiste à plutôt assigner la touche \code{Meta} à
    \code{Commande} (\cmdkey). Cela bloque alors l'accès à certains
    raccourcis Mac, mais la situation est moins critique ainsi.

    Pour assigner la touche \code{Meta} à \code{Commande} (\cmdkey) et
    laisser la touche \code{Option} (\optkey) jouer son rôle usuel, il
    suffit d'insérer les lignes suivantes dans son fichier de
    configuration \code{.emacs} (voir la
    \autoref{emacs+ess:configuration}):
\begin{verbatim}
;;; ====================================
;;;  Assigner la touche Meta à Commande
;;;  et laisser Option être Option
;;; ====================================
(setq-default ns-command-modifier 'meta)
(setq-default ns-option-modifier 'none)
\end{verbatim}
  \end{osx}
  \label{fig:ess:meta}
\end{figure}



\section{Commandes de base}
\label{rstudio:commandes}



\section{Anatomie d'une session de travail (bis)}
\label{rstudio:session}

On reprend ici les étapes d'une \capsule{http://youtu.be/xiNnHegDau8}{session de
  travail} type présentées à la \autoref{presentation:session}, mais
en expliquant comment compléter chacune dans Emacs avec le mode ESS.

\begin{enumerate}
\item Lancer Emacs et ouvrir un fichier de script avec
  \begin{quote}
    \code{C-x C-f}
  \end{quote}
  ou avec le menu
  \begin{quote}
    \code{File|Open file...}
  \end{quote}
  En spécifiant un nom de fichier qui n'existe pas déjà, on se trouve
  à créer un nouveau fichier de script. S'assurer de terminer le nom
  du nouveau fichier par \code{.R} pour que Emacs reconnaisse
  automatiquement qu'il s'agit d'un fichier de script R.
\item Démarrer un processus R à l'intérieur même de Emacs avec
  \begin{quote}
    \code{M-x R }\returnkey
  \end{quote}
  Emacs demandera alors de spécifier de répertoire de travail
  (\emph{starting data directory}). Accepter la valeur par défaut, par
  exemple
  \begin{quote}
    \verb=~/ =\returnkey
  \end{quote}
  ou indiquer un autre dossier. Un éventuel message de Emacs à l'effet
  que le fichier \code{.Rhistory} n'a pas été trouvé est sans
  conséquence et peut être ignoré.
\item Composer le code. Lors de cette étape, on se déplacera souvent
  du fichier de script à la ligne de commande afin d'essayer diverses
  expressions. On exécutera également des parties seulement du code se
  trouvant dans le fichier de script. Les commandes les plus utilisées
  sont alors
  \begin{quote}
    \ess{C-RET}\ pour exécuter une ligne du fichier de script; \\
    \ess{C-c C-c}\ pour exécuter un paragraphe du fichier de script; \\
    \emacs{C-x o}\ pour se déplacer d'une fenêtre à l'autre; \\
    \ess{C-c C-e}\ pour replacer la ligne de commande au bas de la
    fenêtre.
  \end{quote}
\item Sauvegarder le fichier de script:
  \begin{quote}
    \emacs{C-x C-s}
  \end{quote}
  Les quatrième et cinquième caractères de la ligne de mode changent
  de \,\verb|**|\, à \,\verb|--|.
\item Sauvegarder si désiré l'espace de travail de R avec
  \code{save.image()}\indexfonction{save.image}. On le répète, cela
  n'est habituellement pas nécessaire à moins que l'espace de travail
  ne contienne des objets importants ou longs à recréer.
\item Quitter le processus R avec
  \begin{quote}
    \ess{C-c C-q}
  \end{quote}
  Cette commande ESS se chargera de fermer tous les fichiers associés
  au processus R. On peut ensuite quitter Emacs en fermant
  l'application de la manière usuelle.
\end{enumerate}



\section{Configuration de l'éditeur}
\label{rstudio:configuration}



\section{Aide et documentation}
\label{rstudio:aide}


%%% Local Variables:
%%% mode: latex
%%% TeX-master: "introduction_programmation_r"
%%% coding: utf-8
%%% End:

\include{packages}
\chapter{Réponses}
\label{chap:reponses}

\input{reponses-simulation_va}
\input{reponses-arithmetique_ordinateurs}
\input{reponses-resolution_equations}
\input{reponses-revision_algebre_lineaire}
\input{reponses-valeurs_propres}
\input{reponses-decomposition_lu}

%%% Local Variables:
%%% mode: latex
%%% TeX-master: "exercices_methodes_numeriques"
%%% End:
             % NB: différent de Introduction à la...

\bibliography{r,stat,informatique}

\cleardoublepage
\printindex

\cleartoverso

%%% Copyright (C) 2018 Vincent Goulet
%%%
%%% Ce fichier fait partie du projet
%%% «Méthodes numériques en actuariat avec R»
%%% http://github.com/vigou3/methodes-numeriques-en-actuariat
%%%
%%% Cette création est mise à disposition selon le contrat
%%% Attribution-Partage dans les mêmes conditions 4.0
%%% International de Creative Commons.
%%% http://creativecommons.org/licenses/by-sa/4.0/

\vspace*{\fill}

\begingroup
\calccentering{\unitlength}
\begin{adjustwidth*}{\unitlength}{-\unitlength}
  \begin{flushleft}
    \small %
    Ce document a été produit avec le système de mise en page
    {\XeLaTeX}. Le texte principal est en Lucida Bright~OT 11~points,
    les mathématiques en Lucida Bright Math~OT, le code informatique
    en Lucida Grande Mono~DK et les titres en Adobe Myriad~Pro. Des
    icônes proviennent de la police Font~Awesome. Les graphiques ont
    été réalisés avec R.
  \end{flushleft}
\end{adjustwidth*}
\endgroup
\vfill


\cleartoverso

%% Page couverture arrière.
\setkeys{Gin}{width=\paperwidth}
\includepdf[pages=2]{couvertures-partie_1}

\end{document}

%%% Local Variables:
%%% mode: latex
%%% TeX-engine: xetex
%%% TeX-master: t
%%% coding: utf-8
%%% End:
