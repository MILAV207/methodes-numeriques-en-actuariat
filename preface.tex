\chapter*{Introduction}
\addcontentsline{toc}{chapter}{Introduction}
\markboth{Introduction}{Introduction}


Ce document est une collection des exercices distribués par l'auteur
dans ses cours de Méthodes statistiques en actuariat entre 2003 et
2005, cours donnés à l'École d'actuariat de l'Université Laval.
Certains exercices sont le fruit de l'imagination de l'auteur, alors
que plusieurs autres sont des adaptations d'exercices tirés des
ouvrages cités dans la bibliographie.

C'est d'ailleurs afin de ne pas usurper de droits d'auteur que ce
document est publié selon les termes du contrat Paternité-ShareAlike
2.5 Canada de Creative Commons. Il s'agit donc d'un document «libre»
que quiconque peut réutiliser et modifier à sa guise, à condition que
le nouveau document soit publié avec le même contrat.

Le document est separé en deux parties correspondant aux deux sujets
faisant l'objet d'exercices: d'abord la régression linéaire (simple et
multiple), puis les séries chronologiques (lissage, modèles ARMA,
ARIMA et SARIMA). Nous invitons le lecteur à consulter, entre autres,
\citet{AbrahamLedolter}, \citet{DraperSmith} et \citet{BrockwellDavis}
pour d'excellents exposés sur la théorie des modèles de régression et
sur les modèles de séries chronologiques.

L'estimation des paramètres, le calcul de prévisions et l'analyse des
résultats --- aussi bien en régression qu'en analyse de séries
chronologiques --- sont toutes des procédures à forte composante
numérique. Il serait tout à fait artificiel de se restreindre, dans
les exercices, à de petits ensembles de données se prêtant au calcul
manuel. Dans cette optique, plusieurs des exercices de ce recueil
requièrent l'utilisation du système statistique S, soit dans son
incarnation commerciale S-Plus, soit dans sa version libre \textsf{R}.
D'ailleurs, les annexes \ref{chap:regression} et \ref{chap:ts}
présentent les principales fonctions utilisées dans le langage S pour
la régression et l'analyse de séries chronologiques, respectivement.

Le format de ces deux annexes est inspiré de \citet{Goulet_intro_S}:
la présentation des fonctions compte peu d'exemples. Par contre, le
lecteur est invité à lire et exécuter le code informatique des
sections d'exemples \ref{chap:regression:exemples} et
\ref{chap:ts:exemples}.  Le texte des sections d'exemples est
disponible en format électronique dans le site Internet
\begin{center}
  \url{http://vgoulet.act.ulaval.ca/pub/methodes_statistiques/}
\end{center}

L'annexe \ref{chap:elements} contient quelques résultats d'algèbre
matricielle utiles pour résoudre certains exercices.

Tous les jeux de données mentionnés dans ce document sont disponibles
en format électronique à l'adresse
\begin{quote}
  \url{http://vgoulet.act.ulaval.ca/pub/data/}
\end{quote}
Ces jeux de données sont importés dans S-Plus ou \textsf{R} avec l'une
ou l'autre des commandes \texttt{scan} ou
\texttt{read.table}. Certains jeux de données sont également fournis
avec \textsf{R}; la commande
\begin{Sinput}
> data()
\end{Sinput}
en fournit une liste complète.

Nous remercions d'avance les lecteurs qui voudront bien nous faire
part de toute erreur ou omission dans les exercices ou leurs réponses.

Enfin, nous tenons à remercier M.~Michaël Garneau pour sa précieuse
collaboration lors de la préparation de ce document, ainsi que tous
les auxiliaires d'enseignement ayant, au cours des années, contribué à
la rédaction d'exercices et de solutions.

\begin{flushright}
  \itshape
  Vincent Goulet \url{<vincent.goulet@act.ulaval.ca>} \\
  Québec, septembre 2006
\end{flushright}

%%% Local Variables:
%%% mode: latex
%%% TeX-master: "exercices_methodes_statistiques"
%%% End:
