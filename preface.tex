\chapter*{Introduction}
\addcontentsline{toc}{chapter}{Introduction}
\markboth{Introduction}{Introduction}


Ce document est une collection des exercices distribués par l'auteur
dans ses cours de Méthodes numériques en actuariat entre 2005 et 2007,
cours donnés à l'École d'actuariat de l'Université Laval. Certains
exercices sont le fruit de l'imagination de l'auteur, alors que
plusieurs autres sont des adaptations d'exercices tirés des ouvrages
cités dans la bibliographie.

C'est d'ailleurs afin de ne pas usurper de droits d'auteur que ce
document est publié selon les termes du contrat Paternité-Partage des
conditions initiales 2.5 Canada de Creative Commons. Il s'agit donc
d'un document «libre» que quiconque peut réutiliser et modifier à sa
guise, à condition que le nouveau document soit publié avec le même
contrat.

Le cours de Méthodes numériques est séparé en quatre parties plus ou
moins étanches les unes aux autres:
\begin{enumerate}[I.]
\item Introduction à la programmation en S;
\item Simulation stochastique;
\item Analyse numérique;
\item Algèbre linéaire.
\end{enumerate}
Tant la théorie que les exercices de la première partie se trouvent
dans \cite{Goulet_intro_S}. Ce document contient donc les exercices
relatifs aux parties II--IV. Nous invitons le lecteur à consulter,
entre autres, \cite{Ripley_87}, \cite{Gentle_98}, \cite{BurdenFaires}
et \cite{Anton8e} pour d'excellents exposés sur les sujets pré-cités.

Les réponses des exercices se trouvent à la fin de chacun des
chapitres et les solutions complètes en annexe du document.

Nous remercions d'avance les lecteurs qui voudront bien nous faire
part de toute erreur ou omission dans les exercices ou leurs
solutions.

Enfin, nous tenons à remercier MM.~Mathieu Boudreault, Sébastien
Auclair et Louis-Philippe Pouliot pour leur précieuse collaboration
lors de la rédaction des exercices et des solutions.

\begin{flushright}
  \itshape
  Vincent Goulet \url{<vincent.goulet@act.ulaval.ca>} \\
  Québec, décembre 2007
\end{flushright}

%%% Local Variables:
%%% mode: latex
%%% TeX-master: "exercices_methodes_statistiques"
%%% End:
