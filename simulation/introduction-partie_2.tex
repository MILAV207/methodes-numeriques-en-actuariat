\chapter*{Introduction}
\addcontentsline{toc}{chapter}{Introduction}
\markboth{Introduction}{Introduction}

La simulation stochastique est une technique de plus en plus utilisée
en actuariat, mais aussi dans plusieurs disciplines des sciences, du
genre, de la finance, etc. On n’a qu’à penser aux simulations
boursières qui font l’objet d’un concours annuel, aux voitures qui
sont d’abord conçues sur ordinateur et soumises à des tests de
collision virtuels, ou encore aux prévisions météo qui ne sont rien
d'autre que les résultats de simulations de systèmes climatiques
d'une grande complexité.

Toute simulation stochastique repose d’abord et avant tout sur une
source de nombres aléatoires de qualité. Comment en générer un grand
nombre rapidement et, surtout, comment s’assurer que les nombres
produits sont bien aléatoires? C’est un sujet d’une grande importance,
mais également fort complexe. Aussi ne ferons-nous que l'effleurer en
étudiant les techniques de base dans le
chapitre~\ref{chap:generation}.

En actuariat, nous avons habituellement besoin de nombres aléatoires
provenant d’une loi de probabilité non uniforme. Le
chapitre~\ref{chap:simulation} présente quelques algorithmes pour
transformer des nombres aléatoires uniformes en nombres non uniformes.
Évidemment, des outils informatiques sont aujourd'hui disponibles pour
générer facilement et rapidement des nombres aléatoires de diverses
lois de probabilité. Nous passons en revue les fonctionnalités de R et
de Excel à ce chapitre.

Enfin, cette partie du cours se termine au
chapitre~\ref{chap:montecarlo} par une application à première vue
inusitée de la simulation, soit le calcul d’intégrales définies par la
méthode dite Monte Carlo.

L'étude de ce document implique quelques allers-retours entre le texte
et les sections de code informatique présentes dans chaque chapitre.
Les changements dans le fil de la lecture sont clairement indiqués par
des mentions mises en évidence par le symbole {\color{darkred}\noway}.

Chaque chapitre comporte également un lot d'exercices. Les réponses
des exercices se trouvent à la fin de chacun des chapitres et les
solutions complètes en annexe du document.

Je tiens à souligner la précieuse collaboration de MM.~Mathieu
Boudreault, Sébastien Auclair et Louis-Philippe Pouliot lors de la
rédaction des exercices et des solutions.

%%% Local Variables:
%%% mode: latex
%%% TeX-master: "methodes_numeriques-partie_2"
%%% End:
