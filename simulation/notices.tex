\begingroup
\calccentering{\unitlength}
\begin{adjustwidth*}{\unitlength}{-\unitlength}
  \setlength{\parindent}{0pt}
  \setlength{\parskip}{\baselineskip}

  {\textcopyright} {\year} Vincent Goulet \\

  \includegraphics[height=7mm,keepaspectratio=true]{by-sa}\\%
Cette création est mise à disposition selon le contrat
\href{http://creativecommons.org/licenses/by-sa/2.5/ca/}{%
  Paternité-Partage à l'identique 2.5 Canada} de Creative Commons. En
vertu de ce contrat, vous êtes libre de:
\begin{itemize}
\item \textbf{partager} —-- reproduire, distribuer et communiquer
  l'{\oe}uvre;
\item \textbf{remixer} —-- adapter l'{\oe}uvre;
\item utiliser cette {\oe}uvre à des fins commerciales.
\end{itemize}
Selon les conditions suivantes:

\begin{tabularx}{\linewidth}{@{}lX@{}}
  \raisebox{-9mm}[0mm][13mm]{%
    \includegraphics[height=11mm,keepaspectratio=true]{by}} &
  \textbf{Attribution} —-- Vous devez attribuer l'{\oe}uvre de la
  manière indiquée par l'auteur de l'{\oe}uvre ou le titulaire des
  droits (mais pas d'une manière qui suggérerait qu'ils vous
  soutiennent ou
  approuvent votre utilisation de l'{\oe}uvre). \\
  \raisebox{-9mm}{\includegraphics[height=11mm,keepaspectratio=true]{sa}}
  & \textbf{Partage à l'identique} --— Si vous modifiez, transformez
  ou adaptez cette {\oe}uvre, vous n'avez le droit de distribuer votre
  création que sous une licence identique ou similaire à celle-ci.
\end{tabularx}


  \textbf{Code source} \\
  \begin{tabularx}{1.0\linewidth}{@{}Xl@{}}
    Code informatique des sections d'exemples & \href{http://libre.act.ulaval.ca/fileadmin/Portail_libre/ACT-2002/Notes\%20de\%20cours/code-partie_2.zip}{\downloadbutton} \\
    \addlinespace[3pt]
    Code source du document & \href{https://svn.fsg.ulaval.ca/svn-pub/vgoulet/documents/methodes_numeriques/}{\browsebutton}
  \end{tabularx}

  \textbf{Couverture} \\
  Le reptile en couverture est un caméléon de Jackson (\emph{Chamaeleo
    jacksonii}) ou caméléon à trois cornes. On le rencontre en Afrique
  de l'Est, au Kenya et en Tanzanie, ainsi qu'aux États-Unis, plus
  précisément à Hawaï.

  Crédit photo: Michael Nichols, National Geographic Society
\end{adjustwidth*}
\endgroup

%%% Local Variables:
%%% mode: latex
%%% TeX-master: "methodes_numeriques-partie_2"
%%% coding: utf-8
%%% End:
